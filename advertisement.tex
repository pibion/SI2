\documentclass[]{report}   % list options between brackets
\usepackage{parskip}
\usepackage{hyperref}
\hypersetup{
  colorlinks=true,
  linkcolor=blue,
  urlcolor=cyan,
}

% type user-defined commands here

\begin{document}

\title{\TeX\ and \LaTeX}   % type title between braces
\author{Amy Roberts}         % type author(s) between braces
\date{October 27, 1995}    % type date between braces
% \maketitle

%\begin{abstract}
%  A brief introduction to \TeX\ and \LaTeX
%\end{abstract}

\section*{The problem}     % section 1.1

Experimental physics research needs good data on disk.  And DAQ software is both time consuming and, often, entirely custom.  I might spend a month building a useful trigger-rate display tool, but you can't use it - your data format is different.  

Even within an institution, a working DAQ can be alarmingly fragile.  A smooth experiment sometimes depends on access to a single, DAQ-knowledgeable person.  Don't panic, but he's on vacation!  


\section*{The idea}     % section 1.1
The NSF offers a ``Software Infrastructure for Sustained Innovation - SSE \& SSI'' grant
(\url{http://www.nsf.gov/pubs/2016/nsf16532/nsf16532.htm}).  Work on specific software is most appropriate to the SSE grant, which typically awards \$500K over three years and is due in the spring.  SSI grants have a broader scope and typically award \$1M over five years; SSI grants are due in the fall.

I'm interested in writing a grant to (1) define standards for data acquisition and (2) develop and test one to two tools that use those standards.  

I think this proposal is strongest as a joint effort from two groups that use different data acquisition software - such as Midas and the NSCLDAQ, used at Michigan State.

If funded, I imagine supporting two postdocs over three years - one at a CDMS (Midas) institution, and the other at an NSCLDAQ institution.  The work I'd hope to accomplish by the end of the grant would be
\begin{itemize}
 \item develop and publish a core set of DAQ standards
 \begin{itemize}
   \item develop a descriptor language and protocol for streaming data - or adopt an existing one, such as protobuf
   \item \emph{possibly} develop a set of HTTP APIs that allow users to control the DAQ
 \end{itemize}

 \item develop, test, and publish at least one tool based on these standards that works with both Midas and the NSCLDAQ 
 \begin{itemize}
   \item an event builder
   \item \emph{possibly} a scheduler
 \end{itemize}

 \item be able to evaluate if standards work merits further funding through an NSF SSI or S2I2 grant
\end{itemize}

% a google search for "application layer data protocol" was fruitful
% protobuf, a language to define binary data that has bindings to many programming languages
% note that there are javascript bindings for protobuf - and the W3C has a standard for "TextEncoder" and "TextDecoder"
% https://developers.google.com/protocol-buffers/docs/overview

\section*{The benefits}
The proposed work would at minimum result in an event builder, which is required for the completion of the CDMS project.  A performant event builder would also be a significant boon to the beam-based nuclear physics community, as they move increasingly towards digitized readouts.  

Ideally, successful standards extend locally-developed software tools to the entire community. Practically, community-wide standards adoption demands significant time and effort. 

This grant provides some initial manpower: roughly one full-time-equivalent postdoc apiece for CDMS and our NSCLDAQ partner, for three years.  I would expect grant work to split approximately equally between standards work and concrete tool development.



\section*{The odds}     % section 1.1
The success rate for ``Software Infrastructure'' proposals - both SSE and SSI - is about 20\%. SSE grants run for three years, and typically
provide \$500K in total. SSI grants run for five years, and typically provide \$1M
in total.  The S2I2 institute grant provides more, longer-term funding but is not a good fit for this limited-scope proposal.
 
These software grants generally support work that impacts
an entire field; work benefiting a single collaboration rarely receives
funding. %But don't take my word for it!  
I think that our Midas-based group working with an NSCLDAQ partner to explore DAQ standards would be competitive.

Abstracts of funded projects are available at
\url{https://sites.google.com/site/softwarecyberinfrastructure/software/software} and
\url{http://www.nsf.gov/awardsearch/simpleSearchResult?queryText=SI2&ActiveAwards=true}.
 

 
\section*{Worth it?}     % section 1.1
The money is there for the taking, but this is collaborative work at its hardest.  

Nuclear physics demands an impressive breadth from its data acquisition - the standards must be flexible as well as clear and usable.

The full benefits of this project - shareable DAQ tooling - will require years of work.  Meanwhile, building and maintaining community involvement will be critical to broader success.

The software infrastructure grants are here to provide muscle.  Let's build a set of standards for data acquisition, and work together %let the entire community help us 
to get good data on disk.

\section*{Get in touch}
This proposal is in beta.  If you have thoughts, suggestions, or edits of any kind, please get in touch!

You can reach Amy Roberts at amy.l.roberts@usd.edu.


\end{document}
