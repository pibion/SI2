% !TEX root = NSF_SuperCDMS_SNOLAB_OPS.tex

\section{Introduction}\label{sec:overview}

\subsection{The Problem}
The PI proposes improvement and community building for a common set of tools to make it easier for scientists to access their data.

In nuclear physics, in neuroscience, in remote sensing applications, custom data formats are commonplace both for existing and newly-recording datasets.  This fractures the software ecosystem.  Even though the nuclear physics community performs largely similar analyses on largely similar datasets, groups typically develop their software locally.  Poor documentation of these isolated code bases can have real, negative impact on science progress. 
 
%The Data Access Project seeks to build a common set of tools to make it easier for scientists to access custom-format, binary data.

%In many science applications, data is stored in a binary format to make its size manageable - and custom, binary data formats are commonplace both for existing and newly-recording datasets.  This fractures the software ecosystem:  even if a community performs largely similar analyses on largely similar datasets, the custom format leads groups to develop their own, local software. 

%The Data Access Project seeks to increase awareness and adoption of existing data-description languages and to build tools that - given a format description - provide easy access to the information in a data set.

\textbf{S has a data set that would be valuable to include in an analysis} that combines different types of carbon scattering data.  Because the data was taken with a polarized beam, it would provide a valuable constraint to her global fit.  But she can't include it because the only code that can read the data won't compile on her machine.  She's not familiar enough with Fortran to fix it, the original author died a few years ago, and the beamline has been decommissioned.  She could still write a paper, but the results aren't all that interesting without this data. 
 
\textbf{P would like to add an additional detector to his setup} -- this would allow him to set a stronger limit on a reaction rate that effects how stars create heavy elements.  A colleague lent him a digitizer to instrument his extra detector, but the analysis code he has is for CAEN instruments and the loaner digitizer is from XIA.  After burning a weekend trying and failing to adapt his code to handle the XIA data, he decides to try the experiment without the extra detector.  He might get a useful constraint on the integrated cross section, and maybe if the funding for his postdoc comes through they'll have time to sort out the code.
 
\textbf{All Q wants to do is look at the data from a detector used to search for dark matter.}  He got one of the files onto his computer, but when he tried to open it in Word it looked like a bunch of garbage symbols.  Three frustrating days later, he now knows the data is ``binary'' and he's downloaded ``source code''  
that should allow him to look at the data.  When none of the commands in the readme file make sense, he asks his professor for help, who tells him to talk to the postdoc, who apparently doesn't respond to emails.  It takes almost an entire summer, but Q does eventually figure out how to compile the code, run the code, and look at a single event.   

\subsection{Building a Solution}
One way to solve this problem is to define a standard way to describe a data format.  This allows development of tools that provide access to that data based on its description.  Scientists can then keep their custom-format data and – by providing a description of that data – gain access to analysis tools that are supported by a broader community.  The work I propose here works towards this solution in two ways:

The long-term goal of this work is to build and create community buy-in for a set of tools for the nuclear physics community to access binary data.  The intent is to address one piece of a common software need in the community that is repeatedly solved in isolation and that costs the entire community unnecessary time and limits access to data and science. In pursuit of this goal, this proposal has four objectives:  
(1) Define a data-description language that allows scientists to describe any binary data format.
(2) Improve existing software that automatically builds a convenient data-access library for any user-described data so that it works well with gigabyte-scale datasets.
(3) Build community and support adoption of these community tools.
(4) Increase access to analysis of original scientific data sets for undergraduate students, graduate students and postdocs, early career researchers, and scientists whose labs do not have access to dedicated software support. 
These objectives will be met as follows:

{\it (1) Defining a data-description standard:} The PI will use two already-existing standards that both have broad user support, DFDL and Kaitai-Struct.  Kaitai-Struct will be used and where necessary further defined as the software created around this standard is better-suited to the needs of data analysis.

{\it (2) Improving standards-based analysis software:} Scientists who wish to generate a convenient library that allows them to read custom-format data can already do so using the existing Kaitai software.  However, this software has poor performance in python for files larger than a gigabyte and this significantly limits the usefulness of this software to the nuclear physics community, particularly as data acquisition systems move to storing digitized pulses rather than pre-processed data.  The PI proposes to create another Kaitai target that allows scientists to auto-generate a library that targets the awkward-array interface built and maintained by the IRIS-HEP collaboration.  This data structure is designed to give rapid data-analysis access to the high energy physics community and is routinely tested on data sets that exceed several hundred gigabytes.

{\it (3) Building community:} The PI proposes to hold yearly workshops to bring together developers and scientists for training, user feedback, and software development.  In addition, the PI has requested funds for graduate and undergraduate stipends and travel awards to foster community participation.

{\it (4) Increasing Access:} Many students who have the interest and ability to participate in science analysis are blocked by a lack of training in scientific computing.  The often-complex scientific computing ecosystem is poorly represented in most physics curriculum and even computer science courses do not usually prepare students for the variety of systems they need to navigate to do typical data analysis.  The PI requests funding for undergraduate students to develop and test documentation and training materials for fundamental scientific computing concepts and skills in addition to testing the documentation for the standards-based analysis tools.

\subsection{Long-term research goals}


\subsection{The Team}

\comment{This paragraph needs to be added after we do decide on the sections} This project description is organized as follows. 
Sections~\ref{sec:introduction}--\ref{sec:snolab} summarize the science and hardware of the \scs experiment. 
Section~\ref{sec:nexuscute} describes the underground facilities to be used in the pre-operations portion of this proposal.
Section~\ref{sec:operations} describes the plan of work of this effort, and Section~\ref{sec:schedule} describes the schedule.
Section~\ref{sec:management} gives an overview of the management for this proposed grant, and Section~\ref{sec:broad} describes the broader impacts of the work.
Finally, Section~\ref{sec:prev-res} summarizes the results from previous support.

