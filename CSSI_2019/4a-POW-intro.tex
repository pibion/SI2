% !TEX root = NSF_SuperCDMS_SNOLAB_OPS.tex
%% people who build community and knowledge
%% around software?
%% The maintainers
%% Educopia
%% http://ivory.idyll.org/blog/2019-communities-of-effort.html#disqus_thread
%% https://blog.dnanexus.com/2018-01-29-analysis-commons-a-collaborative-approach-to-multi-omics-discovery/

%% describing data
%https://library.si.edu/research/describing-your-data-data-dictionaries
%https://figshare.com/articles/The_State_of_Open_Data_Report_2017/5481187/1
%https://www.usgs.gov/products/data-and-tools/data-management/data-standards
%https://www.fgdc.gov/standards/standards_publications/
%http://www.ddialliance.org/ metadata standards for social sciences
%https://www.rd-alliance.org/groups/data-type-registries-wg.html research data alliance, not focused on physics AFAIK but definitely useful
%https://access-data.trydiscourse.com/



\subsection{Introduction}

\subsection{Science-driven}
% How will the project outcomes fill well-recognized science and engineering needs of the research community, and advance research capability within a significant area or areas of science and engineering? What are the broader impacts of the project, such as benefits to science and engineering communities beyond initial targets, underrepresented communities, and education and workforce development? The project outcomes should address well-recognized science outcomes.

This project serves the immediate needs of researchers in the dark matter community and the experimental nuclear physics community by providing a common toolset for analyzing data in any format.

The PI expects that

\begin{itemize}
    \item Multiple research projects across the NSF directorate will be more productive because they can use existing, documented tools rather than building their own.  The PI intends to estimate this impact with citations from scientific papers.
    \item Increased involvement of undergraduate researchers in science analysis due to improved documentation and an extended support network.  The PI intends to measure this through undergraduate involvement in her own lab, community surveys, and tracking community forum data.
    \item Several example analyses will be publicly released, with accompanying documentation and support information for pre-requisite computing skills.  The intent is for these educational materials to be accessible to someone with no domain knowledge.  The PI believes these training materials will be an equitable training resource.
\end{itemize}


\subsection{Innovation}
% What innovative and transformational capabilities will the project bring to its target communities, and how will the project integrate innovation and discovery into the project activities, such as through empirical research embedded as an integral component of the project activities? Such research might encompass reproducibility, provenance, effectiveness, usability, and adoption of the components, adaptability to new technologies and to changing requirements, and the development of lifecycle processes used in the project.

A common limitation of data-analysis software that is entirely home-grown is that it does not scale as data grows and changes - a human has to update or rewrite the code if the requirements change substantially.

This library makes heavy use of existing libraries.  The benefit is that as those libraries improve and scale to larger data sets, this software inherits that improvement.  

In both cases, significant human effort is needed to adapt the software to changing data and needs.  But by leveraging well-supported, open-source libraries, the burden is shifted away from an individual scientist and towards an active community that is highly motivated to solve similar problems.  This library serves as sugar to allow scientists with many different data formats to take advantage of these popular libraries.

The danger to this approach is the same - if these libraries lose community support or focus on very different problems then this library will lose relevance over time.  To mitigate this risk, the PI is focusing on integrating with a library supported by the IRIS-HEP collaboration and with pandas, which enjoys extreme popularity in the data science community.

\subsection{Close collaboration among stakeholders}

\subsection{Building on existing, recognized capabilities}

\subsection{Project plans, and system and process architecture}

\subsection{Deliverables}


\subsection{Metrics}


\subsection{Sustained and sustainable impacts}