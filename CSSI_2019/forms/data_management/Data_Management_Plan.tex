
\documentclass[11pt]{article}
\usepackage{color}
\textwidth 6.5in
\textheight 9in
\topmargin -.50in
\oddsidemargin 0pt
%\linespread{2.0}
%\renewcommand{\topfraction}{0.5}
%\renewcommand{\bottomfraction}{0.5}
%\renewcommand{\textfraction}{0.1}


\setlength{\parindent}{0pt}
\setlength{\parskip}{1em}

\usepackage{titlesec}% http://ctan.org/pkg/titlesec
\titleformat{\section}%
  [hang]% <shape>
  {\normalfont\bfseries\Large}% <format>
  {}% <label>
  {0pt}% <sep>
  {}% <before code>

\titlelabel{\thetitle.\quad}
\renewcommand{\thesection}{}% Remove section references...
\renewcommand{\thesubsection}{\arabic{subsection}}%... from subsections

\newcommand{\RnD}{{\small R\&D}}
\newcommand{\TES}{{\small TES}}
\newcommand{\SQUID}{{\small SQUID}}
\newcommand{\SQUIDs}{{\small SQUIDs}}
\newcommand{\UCD}{{\small UCD}}
\newcommand{\NIST}{{\small NIST}}
\newcommand{\NEXUS}{{\small NEXUS}}
\newcommand{\BMF}{{\small BMF}}
\newcommand{\SuperCDMS}{{\small SuperCDMS}}
\newcommand{\CDMS}{{\small CDMS}}
\newcommand{\SNOLAB}{{\small SNOLAB}}


\pagestyle{empty}
%\pagestyle{myheadings}%empty}%
%\markright{Budget Justification [\today]}

\begin{document}
%\onecolumn
%\pagenumbering{arabic}
%\setcounter{page}{25}

\section*{Data Management Plan}

\subsection{Expected Data}

The proposed work will result in several software products, specifically

\begin{itemize}
    \item A new kaitai target for the awkward array data structure
    \item An additional release of the XIA python-based analysis library
    \item A prototype release of an analysis library for the Super Cryogenic Dark Matter Search data
\end{itemize}

The proposed work will not generate new data.  Instead, this work will use data already collected.  Small, example datafiles may be prepared for inclusion with the software as test cases.

\subsection{Data Format}

Describe the format in which the data or products are stored (e.g., hardcopy notebook and/or instrument outputs, ASCII, html, jpeg, or other formats). Where data are stored in unusual or not generally accessible formats, explain how the data may be converted to a more accessible format or otherwise made available to interested parties.  You may also comment on the current or anticipated need for interested parties outside of your laboratory to access your primary data.

\subsection{Access to Data and Data Sharing Practices and Policies}

“Access to data” refers to data made accessible without explicit request from the interested party. Examples of this might be data posted on a public website or made available to a public database. Describe your plans, if any, for providing such general access to data, including websites maintained by your research group, and direct contributions to public databases (e.g., IRIS for seismological data, Cambridge Crystallographic Data Centre, Inorganic Crystal Structure Database, the Protein Data Bank, Space Physics Data Center (SPDF), the National Space Science Data Center (NSSDC), Planetary Data System, etc.). Also note if you submit your data in the form of tables, graphs, computer code or other formats to the supplementary materials sections of peer-reviewed journals. Describe your practice or policies regarding the release of data for access, for example, whether data are posted before or after formal publication.

“Data sharing” refers to the release of data in response to a specific request from an interested party. Describe your policies for data sharing, including (if applicable) provisions for protection of intellectual property, national security, or other rights or requirements.


\subsection{Policies for Re-Use, Re-Distribution}

Describe your policies regarding the use of data provided via general access or sharing. For example, if you plan to provide data and images on your website, will the website contain disclaimers, or conditions regarding the use of the data in other publications or products? Describe these disclaimers and/or terms of use.


\subsection{Archiving of Data}

Describe how data will be archived and how preservation of access will be handled. For example, will hardcopy notebooks, instrument outputs, and physical samples be stored in a location where there are safeguards against fire or water damage? Is there a plan to transfer digitized information to new storage media or devised as technological standards or practices change? Will there be an easily accessible index that documents where all archived data are stored and how they can be accessed?  How long will data be retained. 


\end{document}




