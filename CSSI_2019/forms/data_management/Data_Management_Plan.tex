
\documentclass[11pt]{article}
\usepackage{color}
\textwidth 6.5in
\textheight 9in
\topmargin -.50in
\oddsidemargin 0pt
%\linespread{2.0}
%\renewcommand{\topfraction}{0.5}
%\renewcommand{\bottomfraction}{0.5}
%\renewcommand{\textfraction}{0.1}


\setlength{\parindent}{0pt}
\setlength{\parskip}{1em}

\usepackage{titlesec}% http://ctan.org/pkg/titlesec
\titleformat{\section}%
  [hang]% <shape>
  {\normalfont\bfseries\Large}% <format>
  {}% <label>
  {0pt}% <sep>
  {}% <before code>

\titlelabel{\thetitle.\quad}
\renewcommand{\thesection}{}% Remove section references...
\renewcommand{\thesubsection}{\arabic{subsection}}%... from subsections

\newcommand{\RnD}{{\small R\&D}}
\newcommand{\TES}{{\small TES}}
\newcommand{\SQUID}{{\small SQUID}}
\newcommand{\SQUIDs}{{\small SQUIDs}}
\newcommand{\UCD}{{\small UCD}}
\newcommand{\NIST}{{\small NIST}}
\newcommand{\NEXUS}{{\small NEXUS}}
\newcommand{\BMF}{{\small BMF}}
\newcommand{\SuperCDMS}{{\small SuperCDMS}}
\newcommand{\CDMS}{{\small CDMS}}
\newcommand{\SNOLAB}{{\small SNOLAB}}


\pagestyle{empty}
%\pagestyle{myheadings}%empty}%
%\markright{Budget Justification [\today]}

\begin{document}
%\onecolumn
%\pagenumbering{arabic}
%\setcounter{page}{25}

\section*{Data Management Plan}

\subsection{Expected Data}

The proposed work will result in several software products, specifically

\begin{itemize}
    \item A new kaitai target for the awkward array data structure
    \item An additional release of the XIA python-based analysis library
    \item A prototype release of an analysis library for the Super Cryogenic Dark Matter Search data
    \item A release of the Ruby program that allows scanning through binary data
\end{itemize}

The proposed work will not generate new data.  Instead, this work will use data already collected.  Small, example datafiles may be prepared for inclusion with the software as test cases.

\subsection{Data Format}
The software source code and documentation will be stored in ASCII text files.  

Small data files meant to allow testing of the software will be stored in their original, custom binary formats.  The descriptions of these formats will be stored in ASCII text files per the data description standard.

The point of the software developed under this work is to provide easy access to data stored in non-standard formats; documentation for the use of this software will be heavily tested.

%Describe the format in which the data or products are stored (e.g., hardcopy notebook and/or instrument outputs, ASCII, html, jpeg, or other formats). Where data are stored in unusual or not generally accessible formats, explain how the data may be converted to a more accessible format or otherwise made available to interested parties.  You may also comment on the current or anticipated need for interested parties outside of your laboratory to access your primary data.

\subsection{Access to Data and Data Sharing Practices and Policies}
The software created by this project will be publicly available for download from a cloud-based repository host such as github or gitlab.

Additionally, all software products will be registered and archived on Zenodo and be available for download from this website as well.

Software products will be publicly available throughout their development; releases will be used to guide users to stable versions.  Released versions of the software will all be archived and available on Zenodo.

Papers related to the software products will generally be preceded by a software release; the availability of the software products is otherwise independent from publications.

Individuals and organizations who request the software will be able to download the code through the public channels.

The licenses on the software will be as permissible as possible to encourage maximum community adoption.  All code created by the PI's group will be licensed under a permissive open-source license.  In the case of software built with another collaboration the license will need to be negotiated.

Permissive open-source licenses (MIT, Apache, CC-BY 4.0) allow others to re-use the software but does not require that they grant the same license to users of their product.  This makes it easier for companies to use the software since they're not required to share the source code.

Copyleft open source licenses (GPL, BSD) allow others to re-use the software and requires that they make the software available under the same or similar license terms.  Copyleft licenses prioritize keeping source code freely available.  

While the PI finds copyleft a sympathetic aim, this work will only be sustainable and impactful with the broadest possible community adoption.  Companies in this sphere face significant constraints and the PI feels that permissive licenses align best with the adoption goal. 

\begin{itemize}
    \item New Katai target: MIT license
    \item XIA python-based analysis library: MIT license
    \item Analysis library for the Super Cryogenic Dark Matter Search data: still under discussion but likely Apache v3
    \item Ruby program that allows scanning through binary data: GPL 3.0
\end{itemize}

%“Access to data” refers to data made accessible without explicit request from the interested party. Examples of this might be data posted on a public website or made available to a public database. Describe your plans, if any, for providing such general access to data, including websites maintained by your research group, and direct contributions to public databases (e.g., IRIS for seismological data, Cambridge Crystallographic Data Centre, Inorganic Crystal Structure Database, the Protein Data Bank, Space Physics Data Center (SPDF), the National Space Science Data Center (NSSDC), Planetary Data System, etc.). Also note if you submit your data in the form of tables, graphs, computer code or other formats to the supplementary materials sections of peer-reviewed journals. Describe your practice or policies regarding the release of data for access, for example, whether data are posted before or after formal publication.

%“Data sharing” refers to the release of data in response to a specific request from an interested party. Describe your policies for data sharing, including (if applicable) provisions for protection of intellectual property, national security, or other rights or requirements.


\subsection{Policies for Re-Use, Re-Distribution}
Scientists who use the code produced as a result of this work will be asked to cite the version they use using the appropriate Zenodo DOI.

All software downloads will include citation instructions.

Written content on the project website and documentation and any images will be licensed under CC-BY 4.0.

All code will be licensed with an open-source license that allows re-use of the code for commercial purposes.  

Articles written about the software use and development will be published as open access wherever possible.

%Describe your policies regarding the use of data provided via general access or sharing. For example, if you plan to provide data and images on your website, will the website contain disclaimers, or conditions regarding the use of the data in other publications or products? Describe these disclaimers and/or terms of use.


\subsection{Archiving of Data}
All software and documentation will be archived on Zenodo.

Zenodo is a collaboration between CERN and OpenAIRE and has an operation plan for the next twenty years.

Zenodo saves their data on two physically distinct disk servers.  For low-use data, they reserve the right to store the data to tape.

%Describe how data will be archived and how preservation of access will be handled. For example, will hardcopy notebooks, instrument outputs, and physical samples be stored in a location where there are safeguards against fire or water damage? Is there a plan to transfer digitized information to new storage media or devised as technological standards or practices change? Will there be an easily accessible index that documents where all archived data are stored and how they can be accessed?  How long will data be retained. 


\end{document}




