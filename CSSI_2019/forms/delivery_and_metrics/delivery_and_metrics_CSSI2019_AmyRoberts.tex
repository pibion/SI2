
\documentclass[11pt,oneside]{memoir}
%% !TEX root = NSF_SuperCDMS_SNOLAB_OPS.tex
%--------------------------------------------------------------------
%--------------------------------------------------------------------

%%% PACKAGES

%--------------------------------------------------------------------
%--------------------------------------------------------------------

%%% PDF related packages
%\usepackage{lineno}
%\linenumbers
\usepackage{ifpdf} 
\ifpdf % PDF-specific preamble
\usepackage[pdftex,margin=1in]{geometry} % Use the showframe option to draw a ``margin'' box
%\usepackage{showframe}
\usepackage[pdftex]{graphicx}
\usepackage[pdftex]{xcolor} % color or xcolor?
\usepackage[pdftex,colorlinks,urlcolor=blue,linkcolor=blue,citecolor=red,pdfstartview=FitH]{hyperref} %turn this on for clickable PDF files:
%%\pdfinfo{
%%    /Title  (SuperCDMS Base Funding Proposal)
%%    /Author (Tarek Saab)
%%    /Keywords(Saab,Tarek,SuperCDMS)}
\tolerance=600 %\hypersetup{pdfpagemode=UseThumbs}

\else % preamble for LaTeX

\usepackage[dvips]{geometry}
\usepackage[dvips]{color}
\usepackage[dvips,backref]{hyperref}
\fi

\usepackage[final]{pdfpages} %%% Allows you to include pdf files

%----------------------------------------------------------------------
%----------------------------------------------------------------------

%%% Mathematical symbols, figures, floats, etc ...

\usepackage{amssymb}
\usepackage{amsmath}

\usepackage{xspace} %\xspace saves the user from having to type \? or {} after most occurrences of a macro name in text.
\usepackage{isotope} 
%\renewcommand{\isotopestyle}{\sf} 

%\usepackage{subfigure} % make it possible to include more than one captioned figure/table in a single float
\usepackage{wrapfig}
%\usepackage{booktabs} % for much better looking tables
%\usepackage{multirow}
%\usepackage{array} % for better arrays (eg matrices) in maths

%----------------------------------------------------------------------
%----------------------------------------------------------------------

%%% Misc
%\usepackage[light]{draftcopy}

%----------------------------------------------------------------------
%----------------------------------------------------------------------

%%% Appearance

\usepackage{setspace}
\setstretch{1.0}


%%%%%%Call currvita to define the cv environment
%\usepackage[ManyBibs,NoDate]{currvita}
%
%%Define sans serif fonts for headings
%\renewcommand*{\cvheadingfont}{\Large\bfseries\sffamily}
%%\renewcommand*{\cvlistheadingfont}{\large\bfseries\sffamily}
%\renewcommand*{\cvlistheadingfont}{\large\sffamily}

%%\usepackage{bibunits}
%%\usepackage[english]{babel}
\usepackage{paralist} % very flexible & customisable lists (eg. enumerate/itemize, etc.)
%%%%%% These packages are all incorporated in the memoir class to one degree or another...

%%% Pick your font:
%\usepackage{times}
%\usepackage{charter}
%\usepackage{helvet}
%\usepackage{palatino}

%--------------------------------------------------------------------
%--------------------------------------------------------------------

%%% HEADERS & FOOTERS
\usepackage{fancyhdr}
\usepackage[small,compact]{titlesec}
\pagestyle{fancyplain}
\lhead{}
\cfoot{}
\rfoot{\thepage}
\renewcommand{\headrulewidth}{0pt}

%--------------------------------------------------------------------
%--------------------------------------------------------------------

% See the ``Article customize'' template for come common customizations

%%% SECTION TITLE APPEARANCE
%\usepackage[small,compact]{titlesec}   %%%% title sec doesn't play nice with sectsty, i.e. it overrides the all sections font setting. (it it is placed after)

%%\usepackage{sectsty}
%%\allsectionsfont{\sffamily\mdseries\upshape} % (See the fntguide.pdf for font help)

%% chapters, sections, and subsections will have numbers but not subsubsections or below depending on the numerical value
\setcounter{secnumdepth}{3}


%----- Heidi's section macros (From Tali's NU template)
%\newcommand{\mysection}[1]{{\vskip 0 pt \noindent \bf \Large #1\\ \vskip -17 pt }}
\newcommand{\mysection}[1]{{\vskip 0 pt \noindent \bf \large #1\\ \vskip -15 pt }}
\newcommand{\mysubsection}[1]{{\noindent\bf\large #1 \\\vskip -15 pt}}
\newcommand{\mysubsubsection}[1]{{\noindent\bf\em #1}}

% Setting the inter-paragraphs and indent spacing.
\parskip 0 pt
\parindent 20 pt

%--------------------


%%%% ToC APPEARANCE
%\usepackage[none]{tocbibind} % Put the bibliography in the ToC, ... not with the ``none'' options selected.

%%% Can't get this to work with subfigure. Must find out why ???
%\usepackage[titles,subfigure]{tocloft} % Alter the style of the Table of Contents
%\renewcommand{\cftsecfont}{\rmfamily\mdseries\upshape}
%\renewcommand{\cftsecpagefont}{\rmfamily\mdseries\upshape} % No bold!

%%% Float captions and making captions tighter
\usepackage[font={small,sf},labelfont=bf]{caption} % Manipulate caption fonts, make captions smaller, etc...
\renewcommand\floatpagefraction{.8}
\renewcommand\topfraction{.8}
\renewcommand\bottomfraction{.8}
\renewcommand\textfraction{.2}
% and to pack the captions a bit tighter
\floatsep = 10pt  		%separation between floats
%\intextsep=-2pt		%space above and below in-text floats
\abovecaptionskip = 5pt 	%space above caption
\belowcaptionskip = 5pt	%space below caption
\dbltextfloatsep = 0pt 	%space between figure at top of column and text

%----------------------------------------------------------------------
%----------------------------------------------------------------------

%%% If you feel like playing with page layout sizes:
%%\textwidth = 6.5 in
%%\textheight = 8.5 in
%%\textheight = 11 in
%%\addtolength\headsep{0in}
%%%\headsep = 0.5in
%%\addtolength\textheight{-2in}
%%%\addtolength\textheight{-\footskip}
%%\addtolength\textheight{-\headsep}
%%\oddsidemargin =  0.0 in

%----------------------------------------------------------------------
%----------------------------------------------------------------------

%%Define line for front page / appendices
\newcommand{\Line}[0]{
\vspace{-20pt}    %%% this value depends on using the \allsectionsfont{\sffamily\mdseries\upshape}  in sectsty.
\rule{0pt}{0pt}\hrule\rule{0pt}{0pt}
%\vspace{-6pt}
}

%----------------------------------------------------------------------
%----------------------------------------------------------------------

\newcommand{\SkipSection}[2]{
	\ifnum #1 = 1
		% do nothing
	\else 
		% pass on the second argument as is
		#2
	\fi
}


%--------------------------------------------------------------------
%--------------------------------------------------------------------
\usepackage[hang,flushmargin]{footmisc} 

\usepackage{color}
\textwidth 6.5in
\textheight 9in
\topmargin -.50in
\oddsidemargin 0pt
%\linespread{2.0}
%\renewcommand{\topfraction}{0.5}
%\renewcommand{\bottomfraction}{0.5}
%\renewcommand{\textfraction}{0.1}


\newcommand{\fbseries}{\unskip\setBold\aftergroup\unsetBold\aftergroup\ignorespaces}
\makeatletter
\newcommand{\setBoldness}[1]{\def\fake@bold{#1}}
\makeatother

\setlength{\parindent}{0pt}
\setlength{\parskip}{1em}

\setaftersubsecskip{-1em}

\makepagestyle{bodystyle}
%\makeevenhead{bodystyle}{\bfseries\thepage}{}{\scshape
%\MakeLowercase{\leftmark}}
%\makeoddhead{bodystyle}{\scshape\MakeLowercase{\rightmark}}{}{\bfseries\thepage}
\makeevenfoot{bodystyle}{}{}{\bfseries\thepage}
\makeoddfoot{bodystyle}{}{}{\bfseries\thepage}

\newcommand{\RnD}{{\small R\&D}}
\newcommand{\TES}{{\small TES}}
\newcommand{\SQUID}{{\small SQUID}}
\newcommand{\SQUIDs}{{\small SQUIDs}}
\newcommand{\UCD}{{\small UCD}}
\newcommand{\NIST}{{\small NIST}}
\newcommand{\NEXUS}{{\small NEXUS}}
\newcommand{\BMF}{{\small BMF}}
\newcommand{\SuperCDMS}{{\small SuperCDMS}}
\newcommand{\CDMS}{{\small CDMS}}
\newcommand{\SNOLAB}{{\small SNOLAB}}


\pagestyle{empty}
%\pagestyle{myheadings}%empty}%
%\markright{Budget Justification [\today]}

\begin{document}
\mainmatter
\pagestyle{bodystyle}
%\onecolumn
%\pagenumbering{arabic}
%\setcounter{page}{25}

\section*{Delivery Mechanism and Community Usage Metrics}
\subsection*{Elements: Improving tools based on data-description standards for gigabyte-scale data sets}
\hspace{1cm}
\newline
PI: Amy Roberts
%% two page limit

\subsection{Deliverables}

The proposed work will result in several software and documentation products, specifically

\begin{itemize}
    \item Software based on kaitai-struct that interfaces with the data structures developed by IRIS-HEP that are optimized for science data analysis.  This includes end-user and developer documentation.
    \item An additional release of the XIA python-based analysis library that uses the above code for easier analysis of large data sets.  This includes end-user and developer documentation.
    %\item A prototype release of an analysis library for the Super Cryogenic Dark Matter Search data
    \item Releases of ``helper'' libraries that are useful to the scientific community, such as the Kaitai Struct visualizer that reads and nicely displays binary data but needs improvements to work well with gigabyte-scale files.  This includes end-user and developer documentation.
    \item A project website that summarizes and links to the above software and that provides educational resources for fundamental concepts in scientific computing.
\end{itemize}

The proposed work will not generate new data.  Instead, this work will use already-collected data.  Small, example data files may be prepared for inclusion with the software as test cases.

\subsection{Metrics}
The software source code and documentation will be stored in ASCII text files.  

Small data files meant to allow testing of the software will be stored in their original, custom binary formats.  The descriptions of these formats will be stored in ASCII text files following the Katai Struct data description standard.

The point of the software developed under this work is to provide easy access to data stored in non-standard formats; documentation for the use of this software will be heavily tested.

%Describe the format in which the data or products are stored (e.g., hardcopy notebook and/or instrument outputs, ASCII, html, jpeg, or other formats). Where data are stored in unusual or not generally accessible formats, explain how the data may be converted to a more accessible format or otherwise made available to interested parties.  You may also comment on the current or anticipated need for interested parties outside of your laboratory to access your primary data.

\subsection{Access to Data and Data Sharing Practices and Policies}
The software created by this project will be publicly available for download from a cloud-based repository host such as github or gitlab.  Additionally, all software products will be registered, archived, and available for download on Zenodo.

Software products will be publicly available throughout their development; releases will be used to guide users to stable versions.  All releases of the software will all be archived and available on Zenodo.

Papers related to the software products will generally be preceded by a software release; the availability of the software products is otherwise independent from publications.

Individuals and organizations who request the software will be directed to download the code through the public channels.

%The licenses on the software will be as permissible as possible to encourage maximum community adoption.  All code created by the PI's group will be licensed under a permissive open-source license.  In the case of software built with another collaboration the license will need to be negotiated.

Permissive open-source licenses (MIT, Apache, CC-BY 4.0) allow others to re-use the software but does not require that they grant the same license to users of their product.  This makes it easier for companies to use the software since they're not required to share the source code.

Copyleft open-source licenses (GPL, BSD) allow others to re-use the software but requires that they make the software available under the same or similar license terms.  Copyleft licenses prioritize keeping source code freely available.

The PI feels that permissive open-source licenses align best with the goal of broad adoption.  Because this work will only be sustainable and impactful with the broadest possible community adoption, permissive open-source licenses will be preferred wherever possible.  The PI does not anticipate spending NSF resources on closed-source software.

%\begin{itemize}
%    \item New Katai target: MIT license
%    \item XIA python-based analysis library: MIT license
    %\item Analysis library for the Super Cryogenic Dark Matter Search data: still under discussion but likely Apache v3
%    \item Support libraries, if newly developed in the PI's lab: MIT license
%    \item Support libraries, if based on existing open-source projects:  Existing license
%\end{itemize}

%“Access to data” refers to data made accessible without explicit request from the interested party. Examples of this might be data posted on a public website or made available to a public database. Describe your plans, if any, for providing such general access to data, including websites maintained by your research group, and direct contributions to public databases (e.g., IRIS for seismological data, Cambridge Crystallographic Data Centre, Inorganic Crystal Structure Database, the Protein Data Bank, Space Physics Data Center (SPDF), the National Space Science Data Center (NSSDC), Planetary Data System, etc.). Also note if you submit your data in the form of tables, graphs, computer code or other formats to the supplementary materials sections of peer-reviewed journals. Describe your practice or policies regarding the release of data for access, for example, whether data are posted before or after formal publication.

%“Data sharing” refers to the release of data in response to a specific request from an interested party. Describe your policies for data sharing, including (if applicable) provisions for protection of intellectual property, national security, or other rights or requirements.


\subsection{Policies for Re-Use, Re-Distribution}
Scientists who use the code produced as a result of this work will be asked to cite the version they use using the appropriate Zenodo DOI.  All software will include citation instructions in the top-level README file.

The goal of the proposed work is to increase the accessibility of science analysis to the entire community.  Therefore all products will be licensed to allow easy re-use for both non-commercial and commercial purposes:

\begin{itemize}
  \item All written content on the project website and documentation and any images will be licensed under CC-BY 4.0. 
  \item All code will be licensed with an open-source license that allows re-use of the code for commercial purposes.  
  \item Articles written about the software use and development will be published as open access wherever possible; in every case preprints will be published either on the arXiv, figshare, and/or the Open Science Framework.
\end{itemize}

%Describe your policies regarding the use of data provided via general access or sharing. For example, if you plan to provide data and images on your website, will the website contain disclaimers, or conditions regarding the use of the data in other publications or products? Describe these disclaimers and/or terms of use.


\subsection{Archiving of Data}
All software and documentation will be archived on Zenodo.  Zenodo is a collaboration between CERN and OpenAIRE and has an operation plan for the next twenty years.

Zenodo saves all data to two physically distinct disk servers.  For low-use data, they reserve the right to store the data to tape.

%Describe how data will be archived and how preservation of access will be handled. For example, will hardcopy notebooks, instrument outputs, and physical samples be stored in a location where there are safeguards against fire or water damage? Is there a plan to transfer digitized information to new storage media or devised as technological standards or practices change? Will there be an easily accessible index that documents where all archived data are stored and how they can be accessed?  How long will data be retained. 


\end{document}




