\documentclass[11pt,oneside]{memoir}

\usepackage{tabularx}
\usepackage{booktabs,caption}
\usepackage[hidelinks]{hyperref}
\usepackage{refstyle}
\newref{sec}{name=Section~,
             names=Sections~}

\textwidth 6.5in
\textheight 9in
\topmargin -0.50in
\oddsidemargin 0pt
%\linespread{2}
%\renewcommand{\topfraction}{0.5}
%\renewcommand{\bottomfraction}{0.5}
%\renewcommand{\textfraction}{0.1}

\setlength{\parindent}{0pt}
\setlength{\parskip}{1em}

\setaftersubsecskip{-1em}

\makepagestyle{bodystyle}
%\makeevenhead{bodystyle}{\bfseries\thepage}{}{\scshape
%\MakeLowercase{\leftmark}}
%\makeoddhead{bodystyle}{\scshape\MakeLowercase{\rightmark}}{}{\bfseries\thepage}
\makeevenfoot{bodystyle}{}{}{\bfseries\thepage}
\makeoddfoot{bodystyle}{}{}{\bfseries\thepage}

\def\ni{\noindent}
\def\ss{\smallskip}
\def\ms{\medskip}
\def\bs{\bigskip}
\def\eg{{\it e.g.}}
\def\ie{{\it i.e.}}


\newcommand{\cheading}[2]{\textbf{#1\hfill #2}}

%\newcommand{\sec}{Section}
\newcommand{\RnD}{{\small R\&D}}
\newcommand{\ZIP}{{\small ZIP}}
\newcommand{\TES}{{\small TES}}
\newcommand{\CDMS}{{\small CDMS}}
\newcommand{\CDMSI}{{\small CDMS\,I}}
\newcommand{\CDMSII}{{\small CDMS\,II}}
\newcommand{\SNOLAB}{{\small SNOLAB}}
\newcommand{\NEXUS}{{\small NEXUS}}
\newcommand{\CUTE}{{\small CUTE}}

\newcommand{\WIMP}{{\small WIMP}}
\newcommand{\WIMPs}{{\small WIMP}s}
\newcommand{\SuperCDMS}{{\small SuperCDMS}}
\newcommand{\SQUID}{{\small SQUID}}
\newcommand{\SQUIDs}{{\small SQUID\,s}}

%% no more than five pages!
\begin{document}
\mainmatter
\pagestyle{bodystyle}

\section*{Budget Justification}
\subsection{Elements: Improving tools based on data-description standards for gigabyte-scale data sets}
\hspace{1cm}
\newline
PI: Amy Roberts

%{ Budget Years:}
%{ Year 1}:\ 11/1/2019 -- 10/30/2020~~~~~
%{ Year 2}:\ 11/1/2020 -- 10/30/2021~~~~~
%{ Year 3}:\ 11/1/2020 -- 10/30/2022

\section*{A. Senior Personnel}

\subsection*{A1. Principal Investigator (Amy Roberts):}

Assistant Professor Amy Roberts at CU Denver will serve as the Principle Investigator (PI) on this project.  She will commit one week of summer salary per year and and an additional 10\% time in year 3, requested as a course release. Costs for years 2 and 3 include 3\% annual cost-of-living increases. Her responsibilities will include facilitating community adoption of the tools, working with the community to determine the focus of the software development, and guiding and mentoring the staff and students.  In addition, she will supervise one master's student from the Integrated Sciences program.

%The PI has broad experience with data acquisition for medium-scale physics experiments.  She has been a member of the \CDMS\ collaboration since 2014 and has been heavily involved in data acquisition and data quality for the upcoming SNOLAB experiment.  
%Her mentoring in this area provides students and staff with data analysis and acquisition skills that are critical for the field but difficult to obtain.  
%Her expertise with data acquisition, data analysis, and cross-system integration ensure that the SNOLAB data will be science-ready.

The PI is requesting support in excess of two months from the NSF in years 1 and 2 of this proposal. As detailed in the table below, if both the SuperCDMS Operations proposal and this current proposal are fully-funded, the PI would receive 2.15 months of salary support from the NSF from 7/15/2019 to 7/14/2021.  

This support level is justified because (1) CU Denver's standard teaching load is four courses per academic year and (2) CU Denver Physics has neither a Master's nor a Ph.D. program in Physics. Although the PI does have access to Master's students from the Integrated Sciences program, these students specialize in many different fields and cannot always rely on their cohort for technical support. Thus, additional oversight of the group ($\sim$1 scientific programmer, $\sim$1 master's student, $\sim$4 undergraduates) is required during the academic year.  

\begin{minipage}{\linewidth}
    \centering

%\cheading{test}{test}
%\begin{center}
    \begin{tabularx}{\textwidth}{ XXXXX } 
        \toprule
        & 8/15/2018 \newline -~7/15/2019 
        & 7/15/2019 \newline-~7/14/2020 
        & 7/15/2020 \newline-~7/14/2021 
        & 7/15/2021 \newline-~10/30/2022 \\
        \midrule
        \addlinespace[1ex]
        \multicolumn{5}{l}{{{\bfseries SuperCDMS Collaborative Proposal (NSF 1809769)}}}\\ 
        Summer Salary& 0.5 & 0.5 & 0.5 & \textbf{--}\\
        Course Buyout & \textbf{--} & \textbf{--} & \textbf{--} & \textbf{--}\\ 
        \addlinespace[1ex]
        \multicolumn{5}{l}{{{\bfseries SuperCDMS Collaborative Proposal, Operations (Pending, NSF 1809769)}}}\\
        Summer Salary& 0.5 & 0.5 & 0.5 & \textbf{--} \\ 
        Course Buyout& 0.9 & 0.9 & 0.9 & \textbf{--} \\ 
        \addlinespace[1ex]
        \multicolumn{5}{l}{{{\bfseries Current Proposal (Pending, NSF 19-548)}}}\\
        Summer Salary& \textbf{--} & 0.25 & 0.25 & 0.25\\
        Course Buyout & \textbf{--} & \textbf{--} & \textbf{--} & 0.9\\
        \midrule
        Total (months) & 1.9 & 2.15 & 2.15 & 1.15\\
        \bottomrule
      \end{tabularx}
      \captionof*{table}{Table 1.  Reqeusted salary support from all received and pending NSF grant applications, in months.  Note that ``year 1'' in this text refers to the suggested start date of this proposal, 11/1/2019.} \label{tab:title} 
   %\end{center}
\end{minipage}


\section*{B. Other Personnel}

\subsection*{B1. Scientific Programmer:}
The PI requests support for a scientific programmer at 9.6 calendar months in each of the three years of this work with an annual salary of \$42,200. Costs for years 2 and 3 include 3\% annual  cost-of-living increases.  This salary is commensurate with starting salaries for computer science majors in the Denver area.

The PI is confident that she will be able to find an excellent candidate for this position and anticipates hiring a Computer Science and Creative Writing student who has recently joined the group into this position.  This student has extensive programming, writing, and editing expertise.  This combination is  particularly valuable to the work proposed because readable documentation will significantly aid community adoption.

The scientific programmer will work in collaboration with the PI and develop software, software documentation, and work closely with scientists and students in the group to identify issues and solutions for analysis of large data sets with the software.  In addition, the scientific programmer will help coordinate the yearly workshop proposed as a part of this work.

%, at a Year 1 base 1.0 FTE salary of \$57,591

\subsection{B2.  Post Doctoral Scholars:}

No funds for postdoctoral researchers are requested.
%Salary is escalated according to the NIH postdoctoral rate scale in Year 3.

%The postdoc will be recruited at a 1.0 FTE salary of \$47,484 beginning in the second year of the grant.  

\subsection[B3]{B3. Graduate Student:} 
%\seclabel{sec:masters-student-salary}
\label{sec:masters-student-salary}
One Master's student from the Integrated Sciences program or Computer Science program will be employed to assist with development of the software, testing of documentation, and development of example analyses relevant to the nuclear physics community.  This student may also provide computational mentorship to other students in the group.  

Wages are based on \$18/hour in year~1 with projected 3\% annual cost-of-living increases thereafter. Although this rate is low compared to the typical internship rates available to computer science master's students, the PI has had success recruiting excellent students at this wage because the work experience in the PI's lab offers significantly more variety, autonomy, and responsibility than typical corporate internships.  The student will work half-time during the academic year and full-time during the summer.  Please see \hyperref[sec:masters-health]{Section~G1} for discussion of the health insurance fee and \hyperref[sec:masters-tuition]{Section~G1} for tuition support.

\subsection{B4. Undergraduate Students (SA-IV, \$18.00/hr):}

The PI requests funding for an undergraduate researcher at \$18.00/hr working part-time during the semester and full-time during the summer.  

The PI has enjoyed significant success in recruiting highly productive students into the Student Assistants Level IV position, in part because CU Denver has a large population of non-traditional students who already have extensive work experience.
Funding students at a wage that is comparable to market value has been particularly critical at CU Denver, where most students work part- or full-time.  Many students can dedicate significantly more time to research when they are paid for their time.  The PI requests funds at this level to ensure that this all qualified students have access to this research opportunity and vice versa. 

The PI anticipates hiring a current group group member into this role.  This student has been the primary contributor to the successful effort to make the Super Cryogenic Dark Matter Survey (SuperCDMS) analysis environment available to collaboration members through a one-click web interface.  If he is hired, his primary responsibility in year 1 will be testing the standards-based tools on standard SuperCDMS analyses.  The student is expected to graduate in year 2 of this work and the PI requests funding to hire and train students who can eventually take his place.  The primary responsibilities of the new hire in year 3 may shift according to the interests and goals of the student.

\subsection{B5. Undergraduate Students (SA-III, \$12.50/hr):}

The PI requests academic year and summer support for two undergraduate students at \$12.50/hr working 10 hours/week during the semester and 30 hours/week during the summer months.

These students will be employed as research assistants to assist other members of the group with testing of documentation and creation of new documentation where needed.  Depending on their interests and skills, students may be assigned to be the shepherd for a particular analysis and/or to help organize the proposed workshops.  

%Salaries are based on (INSERT DATE) actual salaries and are projected to include a 3~\% annual cost-of-living adjustment effective each year.

\section*{C. Fringe Benefits} 

Fringe benefits are charged as direct costs on all salaries and wages. The rate is 30\% for faculty, 38\% for professional research staff (with 12-month appointments), 19\% for postdocs, and 1\% for enrolled students (academic year) and 1\% for un-enrolled students (summer).

\section*{D. Equipment}
The following equipment (nonexpendable tangible personal property having a useful life of more than one year, and an acquisition cost of \$5,000 or more per unit) is requested in support of this project:

\subsection{D1. Dell PowerEdge Server:}  A PowerEdge server costing \$12,204 (see attached quote) is requested to increase capacity at the Stanford Linear Accelerator (SLAC) web-accessible analysis environment.  This server is identical to servers already in the SLAC pool and can be directly installed into existing infrastructure.  No additional resources are required for the server to function.

This server will support the web-accessible analysis environment, which significantly improves the PI's group's ability to develop, deploy, and test analysis software across the SuperCDMS collaboration.  Interest and use has been growing rapidly and an additional server will ensure that this environment remains stable and available for increased demand.


\section*{E. Travel}

\subsection{E1. Domestic:}
The PI requests a travel budget of \$12,930 (\$4,310/year).  These funds will support yearly attendance at (1) the annual PI meeting and (2) one 3-day conference for the PI and either a student or the scientific programmer.  
%The yearly travel to the PI meeting must be included in this budget per the program solicitation.   
%and attendance at one conference each year provides a valuable chance to disseminate project results and engage with the community served by the software. 
The PI requests conference travel funds in addition to funds for the yearly PI meeting because recruiting community support is critical to the sustainability of this project.  The PI requests conference travel for a student to build their understanding of the community and because the PI has found that a face-to-face meeting often increases communication between remote developers. 

Prospective conferences include the yearly Low Energy Community Meeting (LECM) as well as the International Conference on Computing in High Energy and Nuclear Physics (CHEP) and the Scientific Computing with Python conference (SciPy).  The LECM rotates between nuclear labs in the midwest while CHEP and SciPy - when held in the US - rotate through major cities.  The amount includes airfare, meals/lodging, ground transportation, and registration fees. Estimated breakdown of costs are as follows:

%% heterogeneous DAQs
%https://ieeexplore.ieee.org/document/8204037

\begin{minipage}{\linewidth}
    \centering
    %\begin{center}
    \begin{tabular}{ llllll } 
    \toprule
     Trips & Days & & PI(s) & Student(s) & \\
     \midrule
     1 & 3 & & 1 & 1 &  \\ 
     \addlinespace[1ex]
     Airfare & Meals & Hotel & Transit & Conf. Reg. PI & Conf. Reg. Student\\ 
     \midrule
     \$500 & \$70 & \$120 & \$200 & \$300 & \$200 \\ 
     \bottomrule
   \end{tabular}
%\end{center}
    \captionof*{table}{Table 2.  Itemized domestic travel costs.  The annual PI meeting cost is estimated with these same numbers but with a registration fee of \$0.} \label{tab:title} 
\end{minipage}
    

\subsection{E2. Foreign:}
No funds for foreign travel are requested.  Focusing efforts on the domestic science community provides a compelling and cost-effective scope for this proposal and international scientists interested in collaborating can be supported through teleconferencing.
% Mention ZOOM in facilities and equipment!


\section*{F. Participant Support Costs (20 Participants)} 
Participant support costs are not subject to MTDC.  Because fostering and strengthening community adoption and feedback is critical to the success of the proposed work, the PI plans to recruit up to 20 early-career scientists, late-career scientists, and industry representatives to participate in a yearly 4-day workshop focused on teaching scientists about the tools and using community feedback for rapid development of needed features.  To make the workshop accessible and economical, remote participation will be fully supported; as many as 10 participants are expected to be remote.

\subsection*{F1. Stipends:}  
No funds explicitly allocated to stipends are requested.

\subsection*{F2. Travel:} 
The PI requests \$7,000 for travel scholarships to support early-career researchers and care-giving scientists who wish to attend but need to travel with their dependent.  In addition, these funds may also be used to offset costs for scientists who intend to participate in either development or data analysis and whose participation would bring significant value to the workshop.  Scholarship applicants must write a brief description of their use of the funds.

\subsection*{F3. Subsistence:}  The PI requests \$2,000 for lunches.  Additional food will be paid for through registration fees.

\subsection*{F4. Other:}  The PI requests \$1,000 for the rental of audio-visual equipment and university production services.

\section*{G. Other Direct Costs (Included in MTDC):}
\subsection*{G1. Materials and Supplies:}
We request \$2,000 per year to purchase either an additional workstation or a laptop for software development and testing.  Laptops are included in this budget request because providing a durable, reliable laptop students who face long commutes to campus can significantly increase their time spent on research.

\subsection*{G2. Publication Costs/Documentation/Dissemination}
No funds are requested for dissemination of research.  Hosting the software and community forum through github, gitlab, and discourse is free for academic and open-source use and provide functionality that has allowed many other open source projects to thrive; the PI expects these resources to be adequate for this project as well.  Free, long-term archiving of the software and documentation is available through Zenodo and free hosting of paper preprints is available through arXiv and figshare.  %Thus the PI evaluates that funds for additional publication costs are not required for the success of this project.

\subsection*{G3. Consultant Services:}
No funds for consultant services are requested.

\subsection*{G4. Computer Services:}
No funds are requested for computer services.  While computers in the PI's lab are managed by CU Denver Information Technology, that cost is built into the overhead.

\subsection*{G5. Subawards:}
The proposed work has no subawards.

\subsection*{G6. Other:}
\label{sec:masters-health}
The PI requests funds for the health insurance fee for the Master's student hired for this work.  This cost is \$7,779 over three years.  Please see \hyperref[sec:masters-student-salary]{Section~B3} for discussion of the student assistantship and \hyperref[sec:masters-tuition]{Section~G1} for tuition support.

The PI requests two licenses for a professional vector-graphics program (Adobe Illustrator), an annual cost of approximately \$420. This software is supported on a wide array of devices and is used by students to create documentation that is easy to share and edit.

The PI also requests \$180 per year to pay for a hosted Discourse forum.


%Annual service and maintenance costs on existing software are approximately \$1,000/year. 

\section*{G. Other Direct Costs (not included in MTDC)}
%
%\subsection*{G1. Permanent Equipment:}
%Villano requires equipment for the proposed neutron-capture activities. Specifically, an array of 24 NaI(Tl) gamma detectors has been acquired by CU~Denver for use as the coincident detectors. However, this array requires HV supply photomultiplier bases for each of the detectors, and a digitizer to read them out. The photomultipler bases and digitizer are quoted at \$15,480 total (from Advanced Measurement Technology Inc.) and \$9,998.92 (from CAEN Technologies Inc.), respectively.  

\subsection{G1. Graduate tuition:}
\label{sec:masters-tuition}
The PI requests a total of \$32,568 to pay for the tuition and student fees for a master's student in the Integrated Sciences program for three years.  This amount does not include the cost of student health care; the PI outlines her request for these funds in \hyperref[sec:masters-health]{Section~G6}. This support is in addition to the student assistantship described in 
%\secref{sec:masters-student-salary}~\hyperref[sec:masters-student-salary]{B3}
\hyperref[sec:masters-student-salary]{Section~B3}.

The PI requests tuition, student-assistantship support, and the health insurance fee for a master's student to ensure that this research opportunity remains accessible to all qualified students.  CU Denver has the privilege of serving many non-traditional students, first-generation students, and students who are otherwise under-represented in STEM.  Financial support is often essential for these students to fully participate in the scientific community.

\section*{H. Indirect costs} 

Using DHHS negotiated rates, indirect costs are calculated based on a rate of 55.5\%, applied to a base of Modified Total Direct Costs. 

\end{document}

%Salaries, wages, and benefits:
%	The PI's oversight of the activities will be conducted as part of the normal research component of his academic year salary. In addition, one month of summer salary for the PI is required for additional focused effort during the summer. A full-time Postdoctoral Research Associate is required to conduct the majority of the research year-round. Benefits are included as direct costs.
%	The manpower model in this proposal shifts the bulk of responsibility and task allocation from the PI & many undergraduates to the Postdoctoral Research Associate and Technician. This model has not been attempted at UCDHSC before, but there is a new dedication in the administration to support research activities at a higher level. Moreover, this manpower plan will keep the project running on schedule while continuing the undergraduate research involvement that is the signature of UCDHSC.


%Cost sharing:
%	Cost sharing is not required by program guidelines.
%



