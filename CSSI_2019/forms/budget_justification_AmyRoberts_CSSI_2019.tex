\documentclass[11pt]{article}

\textwidth 6.5in
\textheight 9in
\topmargin -0.50in
\oddsidemargin 0pt
%\linespread{2}
%\renewcommand{\topfraction}{0.5}
%\renewcommand{\bottomfraction}{0.5}
%\renewcommand{\textfraction}{0.1}

\setlength{\parindent}{0pt}
\setlength{\parskip}{1em}

\def\ni{\noindent}
\def\ss{\smallskip}
\def\ms{\medskip}
\def\bs{\bigskip}
\def\eg{{\it e.g.}}
\def\ie{{\it i.e.}}

\newcommand{\RnD}{{\small R\&D}}
\newcommand{\ZIP}{{\small ZIP}}
\newcommand{\TES}{{\small TES}}
\newcommand{\CDMS}{{\small CDMS}}
\newcommand{\CDMSI}{{\small CDMS\,I}}
\newcommand{\CDMSII}{{\small CDMS\,II}}
\newcommand{\SNOLAB}{{\small SNOLAB}}
\newcommand{\NEXUS}{{\small NEXUS}}
\newcommand{\CUTE}{{\small CUTE}}

\newcommand{\WIMP}{{\small WIMP}}
\newcommand{\WIMPs}{{\small WIMP}s}
\newcommand{\SuperCDMS}{{\small SuperCDMS}}
\newcommand{\SQUID}{{\small SQUID}}
\newcommand{\SQUIDs}{{\small SQUID\,s}}


%\pagestyle{myheadings}%empty}%
%\markright{Budget Justification [\today]}

\begin{document}
%\onecolumn

\centerline{\bf Budget Justification}
\centerline{Elements: Improving tools based on data-description standards for gigabyte-scale data sets}
\centerline{PI: Amy Roberts}

%{\bf Budget Years:}
%{\bf Year 1}:\ 11/1/2019 -- 10/30/2020~~~~~
%{\bf Year 2}:\ 11/1/2020 -- 10/30/2021~~~~~
%{\bf Year 3}:\ 11/1/2020 -- 10/30/2022

\section*{A. Senior Personnel}

\subsection*{A1. Principal Investigator (Amy Roberts)}

Assistant Professor Amy Roberts at CU Denver will serve as the Principle Investigator (PI) on this project.  She will commit one summer month per year and and an additional 10\% time in the academic year two, requested as a course release. Her responsibilities will include reaching out to researchers and facilitating their adoption of the tools, working with the community to determine the focus of the software development, and guiding and mentoring the staff and students.  In addition, she will supervise one master's student from the Integrated Sciences program.

The PI has broad experience with data acquisition for medium-scale physics experiments.  She has been a member of the \CDMS\ collaboration since 2014 and has been heavily involved in data acquisition and data quality for the upcoming SNOLAB experiment.  
%She leads a group comprising a Research Technician (to be hired in the upcoming year) and one or more undergraduate research assistants.  
Her mentoring in this area provides students and staff with high-level data acquisition skills that are critical for the field but difficult to obtain.  Her expertise with data acquisition, data quality monitoring, and data analysis and skills with cross-system integration ensure that the SNOLAB data will be science-ready.

One summer month for Roberts is requested in each year of this proposal. Costs for Years 2 and 3 include 3\% annual COL increases. 

Two summer months for Huber are requested in each year of this proposal. In addition, 0.9 academic month support during the academic year (AY) is requested for an academic course release in each year of this proposal. The total support requested for Huber, at 2.9 months (0.9 academic month + 2 summer months) per year, is above the NSF 2-month guideline. This support level is justified due to a combination of CU Denver's standard teaching load of four courses per AY and, more critically, CU Denver Physics having neither a Master's nor a Ph.D. program in Physics. Thus, additional oversight of the large undergraduate component ($\sim$8 undergraduates at all times) is required during the AY. Costs for Years 2 and 3 include 3\% annual cost of living (COL) increases. 

\section*{\bf B. Other Personnel}

\subsection*{B2. Scientific Programmer: Nikki Ramirez}

Bruce Hines will serve as the Senior Engineer for the CU Denver group on this project. He is already hired and has demonstrated excellent qualification for the tasks at hand through past performance on the NSF SNOLAB Project grant. 

Mr. Hines holds a B.S. in Physics and an M.S. in Electrical Engineering and has been a member of the CDMS collaboration since 2005.  He is an experienced design, test, and measurement engineer specializing in all aspects of superconducting electronics, including cryogenics and room temperature electronics. He is a fully qualified machinist and participates in design and fabrication of SQUID test cryostats. As part of the SuperCDMS SNOLAB Project, Mr. Hines supervises students evaluating SQUID arrays and works closely with Fermilab personnel on the room-temperature Detector Control and Readout Card (DCRC). He is the L2 manager of the Readout Wiring and Electronics subsystem.

We request support for Mr. Hines at 4.2 calendar months in Year 1 and at 3.6 calendar months in Years 2 and 3, respectively. Mr. Hines will support commissioning tasks of the room-temperature readout systems at \CUTE, \NEXUS, and \SNOLAB, and will provide operations support of the room-temperature readout systems at \SNOLAB. Costs for Years 2 and 3 include 3\% annual  COL increases. 
%, at a Year 1 base 1.0 FTE salary of \$57,591

\ss \ni 3.  Technical Support (Research Technician): 

Roberts requests 2 calendar months support annually for a Research Technician to contribute to the \SNOLAB\ data acquisition system. The  technician will be responsible for ensuring properly functioning data acquisition systems at the \CUTE\ and \NEXUS\ facilities. Costs for Years 2 and 3 include 3\% annual  COL increases. 

%technician will be recruited at a 1.0 FTE salary of \$35,000.  On this grant, the

\ss \ni 3.  Post Doctoral Scholars (Research Associates): 

Villano requests 6 calendar months support in Years 2 and 3 for a full-time Postdoctoral Research Associate to contribute to the neutron-capture energy calibrations. The postdoc will be responsible for leading neutron-capture calibration efforts at the \NEXUS\ facility.  Salary is escalated according to the NIH postdoctoral rate scale in Year 3.

%The postdoc will be recruited at a 1.0 FTE salary of \$47,484 beginning in the second year of the grant.  

%\ni 3. Graduate Students: 
%
%One Master's level student from Electrical Engineering will be employed to assist with design, test, measurement, and programming of both room-temperature and superconducting electronics. This student may also perform some machine shop duties, and will participate fully in the scientific life of the group.  Wages are based on \$13/hour in Year~1 with projected 3\% annual COL increases thereafter. The student will work half-time during the academic year and full-time during the summer. 
%
%\ss
\ss \ni 3. Undergraduate Students: Huber requests academic year and summer support for two undergraduate students that will be employed as research assistants to assist other members of the group with a variety of tasks, particularly investigation into SQUID sensitivity to ambient magnetic fields and characterization of SQUID 1/f noise. They will also help design, build and operate hardware components, and participate fully in the scientific activities of the group. Wages are based on \$12.50/hour. The total effort will be 5.5 academic months and 4.2 summer months annually.

% and optimization of SQUID design

\section*{C. Fringe Benefits:} 

Fringe benefits are charged as direct costs on all salaries and wages. The rate is 27\% for faculty, 28\% for professional research staff (with 12-month appointments), 19\% for postdocs, 1\% for enrolled students (AY) and 2\% for un-enrolled students (summer).

\section*{D. Equipment:}
The following equipment (nonexpendable tangible personal property having a useful life of more than one year, and an acquisition cost of \$5,000 or more per unit )is requested in support of this project:


\section*{E. Travel: }

\ms
\ni 1. Domestic:  Domestic conferences, \SuperCDMS\ collaboration meetings, and travel for other collaborative research efforts are integral to disseminating the activities of the PI and Co-PIs and supporting the \SuperCDMS\ SNOLAB effort. The total estimated amount of \$71,350 is broken down as follows. \$26,000 is requested to allow the PI, Co-PIs, and scientific and technical staff to attend two collaboration meetings per year at \$1,000 per person per meeting. \$4,350 is requested for inter-institutional travel at three trips at \$1,450 per trip. \$19,500 is requested to allow the PI, Co-PIs, and scientific staff to attend ten domestic conferences at \$1,950 per conference over the award period. \$21,500 is requested to allow the PI, Co-PIs, and scientific staff to cover ten shifts at the NEXUS facility at \$2,150 per shift over the award period.


\ss
\ni 2. Foreign: Major international conferences on dark matter and low-temperature detectors are held regularly, both annually and bi-annually. Funds are requested to allow the PI, Co-PIs, and technical staff to present and disseminate their activities and findings at these significant venues. Additionally, all CU Denver collaboration members cover ``shifts'' at \CUTE\ and \SuperCDMS\ \SNOLAB\ (Canada). The total estimated amount of \$46,010 is broken down as follows. \$21,930 is requested for the PI, Co-PIs, and scientific staff to attend six international conferences over the period of the award at \$3,655 per trip. \$3,290 is requested for the technical staff cover one shift at the \CUTE\ facility. \$20,790 is requested for all CU Denver collaboration members to cover 11 shifts at the \SNOLAB\ facility at \$1,890 per trip over the period of the award.

\section*{F. Participant Support Costs:} 
Participant support costs are not subject to MTDC.

\subsection*{F1. Stipends:}

\subsection*{F2. Travel:}

\subsection*{F3. Subsistence:}

\subsection*{F4. Other:}

\section*{G. Other Direct Costs (Included in MTDC):}
\subsection*{G1. Materials and Supplies:}
We request \$4,500 to purchase, in Year 1, a magnetometer to calibrate Helmholtz coils to be used for \SQUID\ ambient field sensitivity measurements. We request \$7,000 annually to provide liquid helium for performing work on \SQUID\ sensitivity to ambient magnetic fields in existing \SuperCDMS\ \SNOLAB\ \SQUIDs. We estimate costs for general lab supplies, equipment maintenance, and materials at \$1,500 annually. We request \$1,500 in each year for new computers, to gradually phase out aging computers used for instrument control, data collection and analysis, and computational modeling.  All estimates are based on historical usage; the total is \$14,000 in Year 1 and \$10,000 in Years 2 and 3. 

\subsection*{G2. Publication Costs/Documentation/Dissemination:}
%and characterization of 1/f noise 

\subsection*{G3. Consultant Services:}

\subsection*{G4. Computer Services:}

\subsection*{G5. Subawards:}

\subsection*{G6. Other:}

\ss \ni 5. Computer Software: A finite-element computational package (COMSOL) will be used to calculate the interaction between the field coils for sensitivity measurements and the magnetic shielding; the annual service/maintenance contract is approximately \$1,000.

%Annual service and maintenance costs on existing software are approximately \$1,000/year. 

\section*{\bf G. Other Direct Costs (not included in MTDC):}
%
\subsection*{G1. Permanent Equipment:}
Villano requires equipment for the proposed neutron-capture activities. Specifically, an array of 24 NaI(Tl) gamma detectors has been acquired by CU~Denver for use as the coincident detectors. However, this array requires HV supply photomultiplier bases for each of the detectors, and a digitizer to read them out. The photomultipler bases and digitizer are quoted at \$15,480 total (from Advanced Measurement Technology Inc.) and \$9,998.92 (from CAEN Technologies Inc.), respectively.    


\section*{H. Indirect costs: } 

Using DHHS negotiated rates, indirect costs are calculated based on a rate of 55.5\%, applied to a base of Modified Total Direct Costs. 

\end{document}

%Salaries, wages, and benefits:
%	The PI's oversight of the activities will be conducted as part of the normal research component of his academic year salary. In addition, one month of summer salary for the PI is required for additional focused effort during the summer. A full-time Postdoctoral Research Associate is required to conduct the majority of the research year-round. Benefits are included as direct costs.
%	The manpower model in this proposal shifts the bulk of responsibility and task allocation from the PI & many undergraduates to the Postdoctoral Research Associate and Technician. This model has not been attempted at UCDHSC before, but there is a new dedication in the administration to support research activities at a higher level. Moreover, this manpower plan will keep the project running on schedule while continuing the undergraduate research involvement that is the signature of UCDHSC.

%Travel:
%	In order to assist collaborators and test facilities in the proper operation and optimization of SQUIDs, substantial funds for travel are required. Note that most collaboration conferences are held by phone, so the remaining travel represents necessary R&D activities.
%	In addition, funds are requested for the PI & PD to travel to one conference to disseminate the results of the work.

%Cost sharing:
%	Cost sharing is not required by program guidelines.
%



