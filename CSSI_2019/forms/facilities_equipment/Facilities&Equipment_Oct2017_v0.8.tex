
\documentclass[11pt]{article}
\usepackage{color}
\textwidth 6.5in
\textheight 9in
\topmargin -.50in
\oddsidemargin 0pt
%\linespread{2.0}
%\renewcommand{\topfraction}{0.5}
%\renewcommand{\bottomfraction}{0.5}
%\renewcommand{\textfraction}{0.1}


\setlength{\parindent}{0pt}
\setlength{\parskip}{1em}

\def\ni{\noindent}
\def\ss{\smallskip}
\def\ms{\medskip}
\def\bs{\bigskip}
\def\eg{{\it e.g.}}
\def\ie{{\it i.e.}}

\newcommand{\RnD}{{\small R\&D}}
\newcommand{\TES}{{\small TES}}
\newcommand{\SQUID}{{\small SQUID}}
\newcommand{\SQUIDs}{{\small SQUIDs}}
\newcommand{\UCD}{{\small UCD}}
\newcommand{\NIST}{{\small NIST}}
\newcommand{\NEXUS}{{\small NEXUS}}
\newcommand{\BMF}{{\small BMF}}
\newcommand{\SuperCDMS}{{\small SuperCDMS}}
\newcommand{\CDMS}{{\small CDMS}}
\newcommand{\SNOLAB}{{\small SNOLAB}}


\pagestyle{empty}
%\pagestyle{myheadings}%empty}%
%\markright{Budget Justification [\today]}

\begin{document}
%\onecolumn
%\pagenumbering{arabic}
%\setcounter{page}{25}


%\centerline{\bf Facilities, Equipment, and Other Resources}
%\centerline{\bf University of Colorado Denver}
\noindent
\section*{Facilities, Equipment, and Other Resources}

%\section*{University of Colorado Denver}

\bs
%\ni {\bf {\small FACILITIES:}}
\subsection*{\small FACILITIES:}
%\bs
{\bf Laboratory Facilities: }
The PI has dedicated laboratory space on the downtown campus of the University of Colorado Denver, near Physics faculty and staff offices and other Physics Department resources.  Roberts' computational laboratory is 380 sq ft and provides  working space for at least six students.  Her lab includes an ADA-accessible ``telephone booth'' for group members who need to join remote meetings with collaborators.

%This laboratory is equipped for electronics design and prototyping as well as other instrumentation development. It is a ``wet lab'' with two fume hoods, chemical storage areas (for both organics and corrosives), and is set up for three cryogenic test stations. Additionally, Huber's lab is equipped with a large magnetic shield made of a high-$\mu$ material, a large Helmholtz coil for generating DC and AC magnetic fields, and an assortment of vacuum pumps. His lab also includes a small ``hot-work'' brazing area for cryostat fabrication.

%a $^{3}$He cryostat for measurements down to 250~mK,

%\bs
%\ni {\bf Clinical Facilities:}
%\ni N/A
%
%\bs
%\ni {\bf Animal Facilities:}
%\ni N/A

\bs
\ni {\bf Computer Facilities:}
Internet services (off-site as well as connections between the lab and offices) are maintained by the campus Information Technology Services, as are secure web servers for all aspects of the CU Denver campus operations, including research group activities. All offices and laboratories have ample network ports.

\ms\ni
In addition to these campus-level networking resources, our computational tools include dedicated laboratory computers. Roberts' computational tools includes three workstations and three laptops available for student check-out, in addition to monitor stations that are equipped with USB-C docking stations.  This setup allows students to work when and where they need and provides a convenient space for short conferences.  All of the available machines can be used for data analysis and data acquisition development.  All of the available machines have software and operating system support administered by the campus Information Technology Services.

\bs
\ni {\bf Office Facilities:}
\ni The PI is also provided with a private office with telephone and computer network connections. Other personnel will be quartered in laboratory offices as described above or in temporary or shared office spaces.

\bs
\ni {\bf Other Facilities:}
\ni
The PI has access to computing clusters at SLAC and U. of Minnesota.  Access is available to any group members who are performing work relevant to the SuperCDMS collaboration.  The clusters and the test-facility data acquisition computers are maintained by their local institutions. 

%Both Roberts and Villano have active accounts on the clusters listed above and can request computational time for projects related to \CDMS.  In addition Villano will negotiate for time at the \NEXUS\ facility which will have neutron-source and cryogenic detector equipment available for the neutron-capture calibrations he is planning. 
%\bs
\subsection*{\small MAJOR EQUIPMENT:}

The PI does not have additional major equipment; no major equipment is needed for this proposal.
%\ni 
%Huber's group operates a $^{3}$He refrigerator capable of attaining temperatures as low as 250~mK. The experimental volume is approximately 2'' in diameter and 5'' in length. This volume is sufficient for testing superconducting electronics assemblies below 4~K. % and is sufficient for the proposed scanning assembly. The CU Denver group has a set of four vibration isolation pedestals with capacity to support the full scanning system--liquid He Dewar, $^{3}$He cryostat insert, and enclosed scanning stage. 
\bs
\subsection*{\small OTHER RESOURCES:}
\ni
The CU Denver group is provided with department-level support from a shared secretarial staff.   

\ni {\bf Prototyping Facilities:} In addition to dedicated laboratory space, the PI has access to physics department prototyping spaces.  This includes an electronics bench suitable for basic circuit design, construction, and testing; equipped with programmable DC power supplies, programmable signal generators, 50~MHz oscilloscopes, and multimeters.  The prototyping space also has mills, lathes, 3D printers, and a laser cutter for building a variety of apparatus.  A small wood shop is also availble and houses a band saw, jig saw, and hand-held router.  A full-time machinist is available for fabrication and student certification.

{\bf Student Support:}  CU Denver students have 24/7 access to a food pantry that is stocked with non-perishable foods and sanitary products.  In addition, the CU Denver Career Center offers career services to both undergraduate and graduate students and uses the handshake platform extensively to match students with on-campus research opportunities.  The counselors at the Career Center are certified in [some kind of therapy thing] and serve as a complementary career-readiness resource to the PI.
%\medskip
%\ni
%Funding for group members to travel to the test facilities (SLAC, UC Berkeley) and experimental sites (CUTE, NEXUS, and SNOLAB) are being requested by all CU Denver PIs through this grant; Huber has also requested funding for travel through the {\color{red} NSF SuperCDMS \RnD\ proposal and the NSF Pre-Operations and Commissioning proposal}.

%\ms
%\ni 
%Together with {\color{red}parallel \RnD\ and Operations proposals this cycle}, the total support requested for Huber will be 2.9 months annually. Should all proposals be fully funded, this support will be above the 2-month guideline and is justified due to a combination of CU Denver's standard teaching load being four courses per AY and, more critically, CU Denver Physics having neither a Master�s nor a Ph.D. program in Physics, requiring additional oversight of the group by the PI during the academic year. The combined AY year support from the parallel proposals (and thus a reduction in the teaching load) is necessary to allow the PI to cover specific responsibilities associated with these awards. This support above the 2-month guideline does not compensate the PI in the form of salary. Moreover, there is no overlap in support between the {\color{red}existing base and Project awards and the three pending proposals}.
%\medskip

\ni
%All Denver PIs are requesting support for staff.  The existing SuperCDMS SNOLAB project award supports Senior Research Assistant Bruce A. Hines for his work on the SNOLAB project.  The total support from all proposals, should they be fully funded, will result in 0.7 FTE in project year 1, 0.38 FTE in project year 2, and 0.25 FTE in project year 3.  
{\bf Senior Personnel:} Roberts anticipates hiring a professional research assistant (PRA) to work on implementing standards-based data analysis tools.  The Physics Department along with the College of Liberal Arts and Sciences commits additional time for the PRA to work on this proposal.

%  Villano anticipates hiring a 1.0 FTE Postdoctoral Research Associate starting in Year 2 of the requested support.  The total support from this proposal, should it be fully funded, will result in 0.5 FTE in project Years 2 and 3.  

%We are also requesting support from the {\color{red}NSF SuperCDMS \RnD~proposal and the NSF Pre-Operations and Commissioning proposal in each of the three award years}.

%The group also has free access to a liquid nitrogen generator maintained by the Chemistry Department. %The College of Liberal Arts and Sciences provides extensive undergraduate teaching opportunities for instructional staff.  The Department of Physics is committed to providing additional salary and mentored teaching experience for the postdoctoral fellow in addition to the funded research experience.  The University of Colorado Denver has an established Postdoctoral Office to coordinate all aspects of postdoctoral training.

\end{document}




