
\documentclass[11pt]{article}
\usepackage{color}
\textwidth 6.5in
\textheight 9in
\topmargin -.50in
\oddsidemargin 0pt
%\linespread{2.0}
%\renewcommand{\topfraction}{0.5}
%\renewcommand{\bottomfraction}{0.5}
%\renewcommand{\textfraction}{0.1}


\setlength{\parindent}{0pt}
\setlength{\parskip}{1em}

\def\ni{\noindent}
\def\ss{\smallskip}
\def\ms{\medskip}
\def\bs{\bigskip}
\def\eg{{\it e.g.}}
\def\ie{{\it i.e.}}

\newcommand{\RnD}{{\small R\&D}}
\newcommand{\TES}{{\small TES}}
\newcommand{\SQUID}{{\small SQUID}}
\newcommand{\SQUIDs}{{\small SQUIDs}}
\newcommand{\UCD}{{\small UCD}}
\newcommand{\NIST}{{\small NIST}}
\newcommand{\NEXUS}{{\small NEXUS}}
\newcommand{\BMF}{{\small BMF}}
\newcommand{\SuperCDMS}{{\small SuperCDMS}}
\newcommand{\CDMS}{{\small CDMS}}
\newcommand{\SNOLAB}{{\small SNOLAB}}


\pagestyle{empty}
%\pagestyle{myheadings}%empty}%
%\markright{Budget Justification [\today]}

\begin{document}
%\onecolumn
%\pagenumbering{arabic}
%\setcounter{page}{25}


%\centerline{\bf Facilities, Equipment, and Other Resources}
%\centerline{\bf University of Colorado Denver}
\noindent
\section*{Facilities, Equipment, and Other Resources}

%\section*{University of Colorado Denver}

\bs
%\ni {\bf {\small FACILITIES:}}
\subsection*{\small FACILITIES:}
%\bs
{\bf Laboratory Facilities: }
The PI has dedicated laboratory space on the downtown campus of the University of Colorado Denver, near Physics faculty and staff offices and other Physics Department resources.  The PI's computational laboratory is 380 sq ft and provides  working space for at least six students.  Her lab includes an ADA-accessible ``telephone booth'' for group members who need to join remote meetings with collaborators.

%\bs
%\ni {\bf Clinical Facilities:}
%\ni N/A
%
%\bs
%\ni {\bf Animal Facilities:}
%\ni N/A

\bs
\ni {\bf Computer Facilities:}
Internet services (off-site as well as connections between the lab and offices) are maintained by the campus Information Technology Services, as are secure web servers for all aspects of the CU Denver campus operations, including research group activities. All offices and laboratories have ample network ports.

\ms\ni
In addition to these campus-level networking resources, the PI's laboratory has three workstations and three laptops available for student check-out.  The workstations are equipped with USB-C docking stations.  This setup allows students to work when and where they need and provides a convenient space for short conferences.  All of the available machines can be used for data analysis and data acquisition development.  All of the available machines have software and operating system support administered by the campus Information Technology Services.

\bs
\ni {\bf Office Facilities:}
\ni The PI is also provided with a private office with telephone and computer network connections. Other personnel will be quartered in laboratory offices as described above or in shared office spaces.

\bs
\ni {\bf Other Facilities:}
\ni
The PI has access to computing clusters at the Stanford Linear Accelerator (SLAC) and gigabyte-scale dark matter data sets through her collaboration with the Super Cryogenic Dark Matter Search.  These resources are available for testing the analysis tools with large data sets.  The SLAC cluster is maintained by its local institution and the addition of the server requested in this grant will help keep the web-based analysis environment needed for the proposed work stable and available.

%Both Roberts and Villano have active accounts on the clusters listed above and can request computational time for projects related to \CDMS.  In addition Villano will negotiate for time at the \NEXUS\ facility which will have neutron-source and cryogenic detector equipment available for the neutron-capture calibrations he is planning. 
%\bs
\subsection*{\small MAJOR EQUIPMENT:}

The PI does not have additional major equipment; no major equipment is needed for this proposal.

\bs
\subsection*{\small OTHER RESOURCES:}

\ni
{\bf Senior Personnel:} The PI anticipates hiring a professional research assistant (PRA) to work on implementing standards-based data analysis tools, writing supporting documentation and tests, and coordinating the developer side of the proposed community workshops.  %The Physics Department along with the College of Liberal Arts and Sciences commits additional time for the PRA to work on this proposal.

{\bf Administrative Support:}  The PI is provided with department-level support from a shared secretarial staff.  These staff manage time sheets and student pay and help manage travel and purchases.  In addition, the Office of Research Services supports faculty in research endeavors and have staff with expertise in conference and workshop design.   

{\bf Student Support:}  CU Denver students have 24/7 access to a food pantry that is stocked with non-perishable foods and sanitary products.  In addition, the CU Denver Career Center offers career services to both undergraduate and graduate students and uses the handshake platform extensively to match students with on-campus research opportunities.  The counselors at the Career Center serve as a complementary career-readiness resource to the PI.




%  Villano anticipates hiring a 1.0 FTE Postdoctoral Research Associate starting in Year 2 of the requested support.  The total support from this proposal, should it be fully funded, will result in 0.5 FTE in project Years 2 and 3.  

\end{document}




