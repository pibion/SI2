% !TEX root = NSF_SuperCDMS_SNOLAB_OPS.tex
%--------------------------------------------------------------------
%--------------------------------------------------------------------

%%% PACKAGES

%--------------------------------------------------------------------
%--------------------------------------------------------------------
%%% allow comments
%\usepackage{verbatim}

%%% PDF related packages
%\usepackage{lineno}
%\linenumbers
\usepackage{ifpdf} 
\ifpdf % PDF-specific preamble
\usepackage[pdftex,margin=1in]{geometry} 
\usepackage{tabularx}
% Use the showframe option to draw a ``margin'' box
%\usepackage{showframe}
\usepackage[pdftex]{graphicx}
\usepackage[pdftex]{xcolor} % color or xcolor?
\usepackage[pdftex,colorlinks,urlcolor=blue,linkcolor=blue,citecolor=red,pdfstartview=FitH]{hyperref} %turn this on for clickable PDF files:
%%\pdfinfo{
%%    /Title  (SuperCDMS Base Funding Proposal)
%%    /Author (Tarek Saab)
%%    /Keywords(Saab,Tarek,SuperCDMS)}
\tolerance=600 %\hypersetup{pdfpagemode=UseThumbs}

\else % preamble for LaTeX

\usepackage[dvips]{geometry}
\usepackage[dvips]{color}
\usepackage[dvips,backref]{hyperref}
\fi

\usepackage[final]{pdfpages} %%% Allows you to include pdf files

%----------------------------------------------------------------------
%----------------------------------------------------------------------

%%% Mathematical symbols, figures, floats, etc ...

\usepackage{amssymb}
\usepackage{amsmath}
\usepackage{listings}

\usepackage{xspace} %\xspace saves the user from having to type \? or {} after most occurrences of a macro name in text.
\usepackage{isotope} 
%\renewcommand{\isotopestyle}{\sf} 

%\usepackage{subfigure} % make it possible to include more than one captioned figure/table in a single float
\usepackage{wrapfig}


\usepackage{booktabs} % for much better looking tables
%\usepackage{multirow}
%\usepackage{array} % for better arrays (eg matrices) in maths

%----------------------------------------------------------------------
%----------------------------------------------------------------------

%%% Misc
%\usepackage[light]{draftcopy}

%----------------------------------------------------------------------
%----------------------------------------------------------------------

%%% Appearance

\usepackage{setspace}
\setstretch{1.0}


%%%%%%Call currvita to define the cv environment
%\usepackage[ManyBibs,NoDate]{currvita}
%
%%Define sans serif fonts for headings
%\renewcommand*{\cvheadingfont}{\Large\bfseries\sffamily}
%%\renewcommand*{\cvlistheadingfont}{\large\bfseries\sffamily}
%\renewcommand*{\cvlistheadingfont}{\large\sffamily}

%%\usepackage{bibunits}
%%\usepackage[english]{babel}
\usepackage{paralist} % very flexible & customisable lists (eg. enumerate/itemize, etc.)
%%%%%% These packages are all incorporated in the memoir class to one degree or another...

%%% Pick your font:
%\usepackage{times}
%\usepackage{charter}
%\usepackage{helvet}
%\usepackage{palatino}

%--------------------------------------------------------------------
%--------------------------------------------------------------------

%%% HEADERS & FOOTERS
% NSF specifies no headers, footers as of 2018
\usepackage{fancyhdr}
\usepackage[small,compact]{titlesec}
\pagestyle{fancyplain}
\lhead{}
\chead{}
\rhead{}
\cfoot{}
\rfoot{\thepage}
\renewcommand{\headrulewidth}{0pt}

%--------------------------------------------------------------------
%--------------------------------------------------------------------

% See the ``Article customize'' template for come common customizations

%%% SECTION TITLE APPEARANCE
%\usepackage[small,compact]{titlesec}   %%%% title sec doesn't play nice with sectsty, i.e. it overrides the all sections font setting. (it it is placed after)

%%\usepackage{sectsty}
%%\allsectionsfont{\sffamily\mdseries\upshape} % (See the fntguide.pdf for font help)

%% chapters, sections, and subsections will have numbers but not subsubsections or below depending on the numerical value
\setcounter{secnumdepth}{3}


%----- Heidi's section macros (From Tali's NU template)
%\newcommand{\mysection}[1]{{\vskip 0 pt \noindent \bf \Large #1\\ \vskip -17 pt }}
\newcommand{\mysection}[1]{{\vskip 0 pt \noindent \bf \large #1\\ \vskip -15 pt }}
\newcommand{\mysubsection}[1]{{\noindent\bf\large #1 \\\vskip -15 pt}}
\newcommand{\mysubsubsection}[1]{{\noindent\bf\em #1}}

% Setting the inter-paragraphs and indent spacing.
\parskip 0.7 em
\parindent 0 pt

%--------------------


%%%% ToC APPEARANCE
%\usepackage[none]{tocbibind} % Put the bibliography in the ToC, ... not with the ``none'' options selected.

%%% Can't get this to work with subfigure. Must find out why ???
%\usepackage[titles,subfigure]{tocloft} % Alter the style of the Table of Contents
%\renewcommand{\cftsecfont}{\rmfamily\mdseries\upshape}
%\renewcommand{\cftsecpagefont}{\rmfamily\mdseries\upshape} % No bold!

%%% Float captions and making captions tighter
\usepackage[font={small,sf},labelfont=bf]{caption} % Manipulate caption fonts, make captions smaller, etc...
\renewcommand\floatpagefraction{.8}
\renewcommand\topfraction{.8}
\renewcommand\bottomfraction{.8}
\renewcommand\textfraction{.2}
% and to pack the captions a bit tighter
\floatsep = 10pt  		%separation between floats
%\intextsep=-2pt		%space above and below in-text floats
\abovecaptionskip = 5pt 	%space above caption
\belowcaptionskip = 5pt	%space below caption
\dbltextfloatsep = 0pt 	%space between figure at top of column and text

%----------------------------------------------------------------------
%----------------------------------------------------------------------

%%% If you feel like playing with page layout sizes:
%%\textwidth = 6.5 in
%%\textheight = 8.5 in
%%\textheight = 11 in
%%\addtolength\headsep{0in}
%%%\headsep = 0.5in
%%\addtolength\textheight{-2in}
%%%\addtolength\textheight{-\footskip}
%%\addtolength\textheight{-\headsep}
%%\oddsidemargin =  0.0 in

%----------------------------------------------------------------------
%----------------------------------------------------------------------

%%Define line for front page / appendices
\newcommand{\Line}[0]{
\vspace{-20pt}    %%% this value depends on using the \allsectionsfont{\sffamily\mdseries\upshape}  in sectsty.
\rule{0pt}{0pt}\hrule\rule{0pt}{0pt}
%\vspace{-6pt}
}

%----------------------------------------------------------------------
%----------------------------------------------------------------------

\newcommand{\SkipSection}[2]{
	\ifnum #1 = 1
		% do nothing
	\else 
		% pass on the second argument as is
		#2
	\fi
}


%--------------------------------------------------------------------
%--------------------------------------------------------------------