% !TEX root = NSF_SuperCDMS_SNOLAB_OPS.tex

\section{Introduction (1 page)}\label{sec:overview}
\tali{A lot of this text may not be needed as it mirrors the 1-page summary}
The long-term goal of the \scs collaboration is to improve the understanding of dark matter by answering the following questions: what are its constituents, what are their particle and astrophysical properties, and how do they relate to the Standard Model? In pursuit of this goal, this proposal has four objectives:  
(1) Perform necessary calibration and performance measurements of \scs detectors to maximize the science return from the experiment.
(2) Commission and operate the \scs experiment.
(3) Measure a dark matter signal or place world-leading limits for the dark matter--nucleon cross section at masses below 10~\gev.
(4) Train the next generation of scientists through research and outreach activities, and teach secondary students and the public about dark matter science. 
These objectives will be met as follows:

%\comment{Possible rewording for here and the overview page: (3) Train the next generation of scientists to be teacher scholars through engaging them in teaching secondary students and the public about dark matter science. - Joel}
%$\bullet$
%{\it Detector Calibration:} A calibration of the nuclear recoil energy scale is required in order to produce scientific results from the data from \scs. This proposal outlines a calibration measurement campaign which includes high-precision measurements at a neutron beam facility and pre-production \scs detector measurements with a D--D neutron generator at the low-background NEXUS facility located at Fermilab.
 
%{\it Performance and Backgrounds Measurements:} Detailed tests of the performance of \SuperCDMS dark matter detectors must be done in an underground low-background facility. This proposal outlines a measurement campaign of both pre-production \scs detectors and the first production \scs towers at the \cute facility in SNOLAB that will verify the performance of the detectors, measure background levels, provide data to optimize operational parameters, and potentially provide a dark matter search with world-leading sensitivity. 

{\it Experiment Characterization and Optimization:} In order to maximize the science output of \scs, detailed understanding of the response of detectors to potential signals and backgrounds is required. Measurements at two underground facilities (\nexus at Fermilab and \cute at SNOLAB) will characterize and calibrate pre-production \scs detectors and the first production \scs towers, with supporting measurements from small devices at university test facilities. This data will allow energy scale calibration, measurements of background levels, understanding of the detailed phonon and electron physics of the detectors, and optimization of \scs operational parameters. 

{\it Commissioning and Operation of \scs:} This proposal will fund the NSF-portion of commissioning the \scs experiment and the first year of science operations.

{\it First \scs Science:} This proposal will perform a dark matter search with  world-leading sensitivity using data taken with one \scs Tower at \cute and with the first data set from the full \scs experiment.

{\it Training and Outreach:} The activities in this proposal provide fertile ground for training of postdoctoral researchers and graduate students from the \SuperCDMS collaboration, and opportunities to share the excitement of science with the general public.

\textbf{Intellectual merit}: This grant supports an experiment that will address one of the most fundamental problems of modern science, the nature of dark matter, in a way complementary to the Large Hadron Collider and indirect detection experiments. The \scs experiment will achieve world-leading sensitivity for dark matter  masses less than 10~\gev.

\textbf{Broader impacts}: The \scs experiment will have a broad impact which extends beyond the search for dark matter. The experiment's phonon-mediated detectors have applications in cosmology, astronomy and industry. This effort will contribute opportunities for training of undergraduate and graduate students and postdoctoral researchers.

SuperCDMS will engage in education and outreach at local institutions and will coordinate our outreach at Soudan, SNOLAB, and SURF to increase the broader impact with focus on secondary education. The grant will support participating in SNOLAB-hosted Teacher Workshops, modifying and providing middle school dark matter education modules to U.S. educators, and maintaining SuperCDMS-related outreach materials for SNOLAB, Soudan, and SURF. Existing alliances will enable outreach to Sudbury First Nations communities. This effort will also support post-graduate education by providing speakers for Colegio de F\'{i}sica Fundamental e Interdiciplinaria de las \'{A}mericas summer school in Puerto Rico.

\comment{This paragraph needs updating: }The activities in this proposal are fully coordinated with the \SuperCDMS\ collaboration in general, and its DOE and Canadian-supported institutions. This NSF proposals is tightly coordinated with two R\&D proposals titled ``\SuperCDMS R\&D Toward the Neutrino Floor" that have been awarded from the DOE KA23 and KA25 programs by Fermilab and SLAC respectively. This proposal is also coordinated with the DOE \scs Experimental Operations Plan.  

This collaborative proposal subsumes most of the work done by the NSF institutions on \scs. This includes all calibration, characterization, commissioning, operation, analysis, and publication efforts. It also includes support for students, postdocs, and scientists that would have normally been funded through separate ``base'' proposals by each member institution. This support and expertise from university groups will be instrumental in turning the data products from the \scs experiment into fully vetted scientific results and publications. In addition, this proposal provides fertile ground for training graduate students and postdocs through their participation in the data taking and analysis. 

\comment{This paragraph needs to be added after we do decide on the sections} This project description is organized as follows. 
Sections~\ref{sec:introduction}--\ref{sec:snolab} summarize the science and hardware of the \scs experiment. 
Section~\ref{sec:nexuscute} describes the underground facilities to be used in the pre-operations portion of this proposal.
Section~\ref{sec:operations} describes the plan of work of this effort, and Section~\ref{sec:schedule} describes the schedule.
Section~\ref{sec:management} gives an overview of the management for this proposed grant, and Section~\ref{sec:broad} describes the broader impacts of the work.
Finally, Section~\ref{sec:prev-res} summarizes the results from previous support.


\subsection{Introduction}
\label{sec:introduction}

\tali{This should be trimmed and incorporated into either the project overview or the science section}
A critical research area in particle astrophysics is the search for the nature of dark matter. Dark matter, unlike ordinary matter, does not produce electromagnetic radiation and its only observed interaction with ordinary matter thus far has been gravitational. Its existence explains flat rotation curves of spiral galaxies, structure formation and galaxy cluster evolution, and the anisotropy of the cosmic microwave background. These astronomical observations indicate that approximately 85\% of the matter in the universe must be dark matter, but they do not tell us much about its composition. A long-favored hypothesis is that a class of elementary particles generically labeled as WIMPs (Weakly Interacting Massive Particles) constitutes the dark matter. However, an absence of evidence for a heavy mass WIMP at the LHC and in direct detection dark matter experiments has motivated the development of a wide array of alternate theories which predict dark matter particles with masses between 1~MeV--10~GeV. 

Determining the fundamental nature of dark matter is a high priority science objective for the National Science Foundation (NSF) and the DOE Office of High Energy Physics (OHEP). It is also a major focus for science in Canada, with support from the Natural Sciences and Engineering Research Council of Canada (NSERC) and the Canada Foundation for Innovation (CFI). 

The SuperCDMS collaboration has conducted a series of experiments designed to directly detect dark matter in underground laboratories, using ultra-pure germanium (Ge) and silicon (Si) crystals outfitted with ionization and phonon sensors and operated at cryogenic temperatures. For most of the last two decades, CDMS experiments at Stanford and Soudan have provided leading sensitivity to high-mass WIMPs. Recently SuperCDMS has focused on $<$~10~\gev dark matter candidates, where our technology has unique advantages given the low energy thresholds provided by their athermal phonon sensor technology and Luke-Neganov phonon amplification techniques for ionization.

The NSF and DOE have jointly chosen SuperCDMS SNOLAB as the next-generation project to search for low-mass dark matter. The project is going through the DOE Critical Design (CD) review process and is expected to be complete in 2019. Canada is also supporting SuperCDMS SNOLAB via NSERC and CFI funding. The experiment will be located at SNOLAB, in Ontario, Canada, the deepest underground laboratory in North America.
