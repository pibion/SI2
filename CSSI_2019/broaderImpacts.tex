% !TEX root = NSF_SuperCDMS_SNOLAB_OPS.tex
\section{Broader Impacts}
\label{sec:broad}

The SuperCDMS collaboration continues to deliver on its potential for broader impact including strong technical development, education at all levels, and engaging public outreach.

\textbf{Technical Development}
CDMS and SuperCDMS have developed, and continue to advance detector technologies that have significant impact both inside and outside of the dark matter community. These technologies include electrothermal feedback, transition-edge sensors (TESs); detection and utilization of athermal phonons; large, kilogram-scale phonon-mediated germanium and silicon-based detectors; arrays of large-scale, low-noise SQUIDs; and usage of LPN HEMPTs for ionization readout.

The field of direct detection of dark matter has benefited from these advances. The EDELWEISS and CRESST collaborations have both implemented some of these concepts. In addition, SuperCDMS’ pioneering of very-low-mass WIMP detection has had a strong impact on the field. There are now many R\&D efforts focused on very-low-mass WIMP detection. Multiple of these programs are run by PIs who were primarily engaged in liquid noble direct detection. The concepts developed by SuperCDMS have had far reaching impacts benefiting even direct detection experiments using the liquid nobles. 

The broader, basic-science community has benefited from SuperCDMS technology advances. The TES technology is now a core technology for multiple cosmic microwave background experiments including APEX, South Pole Telescope, and PolarBear. It is being rapidly adapted for far-infrared instruments including Super-SCUBA, single-photon spectroscopy, and large X-ray calorimetry experiments at NIST and NASA GSFC including the forthcoming Micro-X, which will launch the first TES into space. In addition, the TES technology is beginning to impact industry as TESs are being adapted for use in electron microscope sensitive surface element analysis and are being investigated for use in quantum coherence and computing.

The direct impact of SuperCDMS research is underscored by the number of graduate students impacted and the number and quality of the SuperCDMS publications. In the past three years, SuperCDMS has produced 12 Ph.D. theses and 11 science publications not including conference proceedings. The impact of these publications is underscored by their many citations (over 3,000 since 2011).

\textbf{Undergraduate, Graduate, and Postgraduate Education}
Direct detection of dark matter is multidisciplinary and therefore creates a broad spectrum of opportunities for students to explore and advance in many fields. From year to year, numerous undergraduates, graduate students and postdoctoral researchers from the SuperCDMS NSF-funded institutions have benefited from these collaborative opportunities and have been guided to develop precise simulations of semiconductor-based detectors, sensitive analysis techniques, statistical applications, and application of advanced software techniques. These opportunities combined with regular mentoring have afforded many high-visibility talks for students and postdocs. 

As a collaboration, SuperCDMS strives to ensure that graduate students and postdoctoral researchers receive proper exposure to the wider dark matter and particle physics communities to support their scientific development and establish the basis for future career and collaborative opportunities. To that end, a committee within the collaboration, headed by a senior member, identifies and advocates for high profile talk opportunities within the dark matter field. PIs from institutions on this proposal have provided strong representation on this committee. Interactions between university-based scientists and technical experts in industry, particularly in the areas of cryogenics and electronics continue to provide long-term benefits to the physics community at large, and are likely to lead to excellent job opportunities for some of our students.

Over the past seven years, SuperCDMS grad students have attained postdoctoral appointments at Caltech, Stanford, MIT, UC Santa Barbara, Texas A\&M, Northwestern, NASA GSFC, MIT, Berkeley, Lincoln Labs, LBNL and Fermilab, and others are pursuing careers in industry and tech. Also in this period, grad students and postdoctoral researchers from SuperCDMS NSF-funded institutions have obtained faculty appointments at Caltech, UC Berkeley, U. Illinois at Urbana-Champaign, San Diego State U., U. Massachusetts-Amherst, and three are senior researchers at NIST, SNOLAB and PNNL. The current SuperCDMS analysis coordinator is from an NSF-funded institution. 

Broad engagement of undergraduates in the research continues to be a strong focus. Currently, there are more than 25 undergraduates involved in SuperCDMS research, gaining the direct lab experience that is increasingly a discriminator in graduate school admission and an important complement to the undergraduate, mostly theoretical, education. At Northwestern University, the PI participates in the Summer Research Opportunities Program, which brings students from other US institutions to Northwestern and other universities in the Big Ten Alliance. Of the 34 undergraduates he has supervised, 15 were female and 10 were members of an underrepresented minority. 

Colegio de Física Fundamental e Interdiciplinaria de las Ámericas (COFI) is a physics research institute in San Juan, Puerto Rico, which hosts summer schools in different topics in physics, for graduate and undergraduate students, as well as talks for public audiences. In Summer 2016, the topic was high-energy physics instrumentation, in which Northwestern University PI Figueroa-Feliciano participated. The Institute is interested in partnering with SuperCDMS to bring theorists and experimentalists to their summer schools to give talks on dark matter and related topics. COFI opens its summer schools to all students (undergraduate and graduate students), but has a particular mission of bridging the span between North America and South America, bringing students and researchers from South America to interact with US researchers in San Juan. Given its unique character as a Spanish-speaking location in the US, this is a natural bridge between the communities. COFI is also interested in reaching students from underrepresented groups and bringing them to their summer schools.  In year 2, SuperCDMS will provide two speakers on dark matter-related topics at a COFI summer school. 

At Berkeley, undergraduates will be largely drawn from subsidized campus programs, including the Undergraduate Research Apprenticeship Program (URAP) and CalTeach, a program that enables science majors to concurrently complete their BS degree and a California teaching credential. The Compass Project, an initiative developed by Berkeley physics graduate students and launched in August 2007, aims to increase the percentage of minority and women students that matriculate with a bachelors degree in one of the physical sciences by engaging students who express an interest in such fields as early as possible. As the program’s faculty advisor, Bernard Sadoulet has facilitated their efforts to ensure the sustainability of this proven recruitment and retention program, by helping them secure a long-term, institutionalized base of support. Compass received the 2012 Award for Improving Undergraduate Physics Education, by the American Physical Society. Compass offers an intensive summer program for incoming freshmen from underserved schools, fall and spring term retention courses (including a course for transfer students to support their transition to Berkeley), mentoring, a research lecture series, and other social and academic support. %(http://www.berkeleycompassproject.org/programs/physics-98/).

Sadoulet is the director of Berkeley Connect in Physics, a mentoring program within the physics department that accepts undergraduate students at all levels. The course is a small seminar class led by a physics graduate student, under the guidance of the director. The goals of the program are to help students develop understanding, community, and career preparedness that go beyond what traditional courses provide. Interactions with graduate students and faculty play a large role throughout the semester. Under Sadoulet’s leadership, this course has significantly gained in enrollment with every succeeding semester since its launch in Spring 2014. %(http://www.berkeleyconnect.berkeley.edu/physics/).

The Physics Department at CU~Denver maintains an undergraduate-only program committed to providing students with hands-on research experience at this urban campus. The Department operates in close collaboration with the Physics Department at Metropolitan State University of Denver, and students from both institutions participate in the CU~Denver group’s activities, broadening the reach of the research program beyond a single institution. Underrepresented groups and first-generation college students are well represented in the PI’s group, and many of his mentees have proceeded to graduate school in physics and other disciplines, or embarked on careers in industry at companies such as Northrop Grumman. Currently, there are eight undergraduates in the CU~Denver group majoring in Physics, Electrical Engineering, and Mechanical Engineering.  

At the U. Florida, the PI has been the Education Outreach contact person for the Physics Department. His group hosts three (non UF) undergraduates for summer research internships as part of the REU program. 

%The TAMU SuperCDMS group has four PIs who oversee the activities of four postdocs, eight graduate students, and more than six undergraduate students who contribute to the research, including detector fabrication and polishing efforts. 

Southern Methodist U. typically funds one to three undergraduate students to work with the Cooley group each year. In the past five years, the eight undergraduates who have participated in the group’s research have either entered graduate programs in physics or engineering or embarked on careers in industry at companies such as Texas Instruments and AT\&T.  

The PI at the U. South Dakota serves as the Chair of the SuperCDMS Outreach Committee, and is a member of the Education and Outreach committee at USD. He increases the opportunities for undergraduates at USD and other institutions by mentoring non-physics and physics undergraduates in their honors thesis. The PI has been instrumental in bringing new, popular physics demonstrations to life including a Ruben’s tube and a cloud chamber that are now used in introductory physics classes. He has recruited four undergraduates who have contributed to the PI’s research program in the past two years, and he has guided two undergraduates through the process of writing a proposal for a NASA Research Stipend. A major issue confronting many undergraduate physics majors from colleges and smaller universities is the lack of exposure to the many fields of physics research. To address this issue, the PI presents colloquia and seminars at local colleges and universities, helping to spread the word about the new USD PhD program. All domestic students accepted into the USD Fall 2015 graduate class were a direct result of the PI’s talks. The PI is attempting to strengthen undergraduate research opportunities at USD through partnership on two USD NSF REU proposals. The PI was an organizer of the 2015 NSF sponsored USD Germanium Workshop and actively sought to increase minority and female participation at the workshop. As a result, 23\% of the participants were women, 80\% of whom were presenters. The PI personally invited one third of the minority attendees and arranged for all of them to give plenary talks. 
In addition to engaging undergraduates on campus, SuperCDMS will support an 8-week undergraduate summer research internship at SNOLAB, in years 2 and 3.

\textbf{K-12 Education}
SuperCDMS institutions provide strong support for K-12 education through programs for both teachers and students. Too numerous to describe in detail in this proposal, these programs take place in local schools, on campus and directly in the lab. At the core of many of the programs is engagement of students from disadvantaged backgrounds and underrepresented groups, including women. 

CU Denver, U. Florida, SMU, and USD engage middle school students in physics learning through lab tours, classroom visits and science camp. Northwestern U., UC Berkeley, U.~Florida, TAMU, SMU, and SD School of Mines and Technology support a range of high school programs, including summer research internships for students and teachers, hands-on research projects for underserved high school students of color, mini-courses to prepare Native American students for postsecondary education in science and engineering, interviews with physicists conducted by high school students, and structured physics courses for high school teachers. 

\textbf{Local Public Outreach}
SuperCDMS groups are actively engaged in efforts to share SuperCDMS science with the public. PIs and group members at Northwestern, UC Berkeley, and CU Denver present public lectures on dark matter, on campus, at community events and civic organizations, and at science cafes. The PI at Northwestern has presented two talks in the past year in San Juan, Puerto Rico. 

UC Berkeley, U. Florida, SMU, SD School of Mines and Technology, and TAMU participate in outreach events on campus, at community festivals, local farmers’ markets, science fairs and science camps. Activities include lab tours and hands-on physics demonstrations. SuperCDMS scientists have been featured in the media, including NPR’s Science Friday and in the popular press. 

\textbf{Site-Based SuperCDMS Education \& Outreach}
In recent years, SuperCDMS has taken advantage of the Soudan Underground Laboratory’s (SUL) location in a state park to impact thousands of tourists and dozens of school groups each year through laboratory science tours. These tours have taken students and tourists 2,341ft below the surface to the underground laboratory to view the SuperCDMS and MINOS experiments. At a SuperCDMS interactive table on the tours, tourists and students had the opportunity to not just see and hear about the SuperCDMS experiment but also physically interact with experimental elements including a sample detector. In addition, SuperCDMS scientists would speak directly with the public at a full-day science Open House, which attracted around 600 people annually. At the surface, a touch-screen kiosk continues to provide videos and descriptions of the SuperCDMS (and MINOS) experiments. 

With the departure of SuperCDMS from SUL and the completion of MINOS, underground outreach activities have ceased. However, the Ash River SUL group, under the leadership of Prof. Richard Gran (U. Minnesota-Duluth) is continuing outreach activities at a nearby NOvA site, with plans for a NOvA-centric summer program. The SUL Outreach Education Coordinator reports that public audiences and science educators are intensely interested in SuperCDMS science, and staying current with SuperCDMS SNOLAB is a priority. SuperCDMS will team with the Ash River SUL group (see the attached letter of collaboration) as a perfect conduit to continue our SUL Minnesota outreach. We will provide the Ash River SUL group SuperCDMS support and materials including posters, and virtual tours focused on the history of dark matter and neutrino physics. SuperCDMS will also enable the NOvA SUL group to effectively speak about dark matter by continuing to develop answers to commonly asked dark matter questions.

SuperCDMS is already reaching out to the public at the future experimental site, SNOLAB, and envisions growing that outreach during the timescale of this proposal to a level similar to the Soudan outreach. SuperCDMS is currently engaging SNOLAB tour groups by displaying a poster at a regular tour stop in the underground lab. In addition, SuperCDMS is highlighted in the SNOLAB “NewEyes on the Universe,” an interactive, mobile exhibit, which debuted in London, England, in July 2016.

SuperCDMS’s SNOLAB education and outreach will ramp up on a timescale parallel to its operational ramp. SuperCDMS will increase the impact on underground tours by developing an interactive display that enables tour members to explore the question of dark matter and the SuperCDMS experiment. SNOLAB has developed an online virtual tour that enables the general public to tour the SNOLAB underground lab remotely, be it from home, school, museum or community center. 

The “Ladder Labs” area, which will house SuperCDMS, is already featured in the virtual tour. SuperCDMS will explore ways to strengthen its engagement with virtual tourists including virtual science “treasure hunts.” These hunts will be designed to tell the science stories behind SuperCDMS’s search for dark matter. For example, one hunt will follow the life story of a dark matter detector as it is built at TAMU, tested at Berkeley, and installed at SNOLAB. Another will tell the story of the importance of eliminating radioactive backgrounds and how the SMU and SDM\&T groups are helping to solve this problem for SNOLAB. These virtual tours will be augmented by a web presence and social media platforms, making the science compelling and accessible to teachers, students and the general public.

We will coordinate our outreach efforts with the SNOLAB education and outreach team to leverage resources. For example, each year, high school students enroll in the International Summer School for Young Physicists. The students spend part of their time on underground science tours at SNOLAB. SuperCDMS graduate students and postdocs will serve as science guides for these students during the time underground. Each summer, the Tri-Institute Summer School on Elementary Particles (TRISEP) is held at Laurentian University in Sudbury. We will participate by sending graduate students to TRISEP and offering SuperCDMS faculty to serve as TRISEP educators. 

The 2017 TAUP conference will be held in Sudbury, and we anticipate using TAUP as a launching point for faculty and student involvement at SNOLAB. In addition, we will participate in SNOLAB education and outreach programs as opportunities arise and will provide outreach materials to SNOLAB.

SuperCDMS will broaden its scope of outreach to also include the Sanford Underground Research Facility (SURF). At SURF, the USD and SDSM\&T PIs are working with the SURF E\&O team to coordinate an outreach program beginning with a presence at SURF Neutrino Day. Neutrino Day is an annual outreach event at SURF that has live science demonstrations and talks that directly reaches about 1,000 people and indirectly reaches tens of thousands through resulting media coverage including a live South Dakota Public Radio broadcast.

In addition, we will develop six dark matter education modules for integration of SuperCDMS science into the SURF “Search for Dark Matter” middle school curriculum. These units are designed to aid teachers in the classroom and fill a South Dakota educational requirement. It is part of SURF’s strategic plan to expand these beyond South Dakota. The Berkeley group will look into how three of these modules can be modified to accommodate California educational standards, and the Northwestern U. group is willing to host three of these modules for use by educators in the Chicagoland area. We will support a SURF education person to travel to Sudbury to provide teachers with a 2 to 3 day professional development workshop on the implementation of the curriculum materials. 

Another priority is to establish an alliance with Sudbury's First Nations communities for the sharing of science education. This is of interest to the outreach division at SNOLAB, and SuperCDMS has already expressed interest in working with them on development. CDMS at Berkeley has experience in this area through an established association with the founder and leadership team of the Native American Science Academy. Through this alliance, SuperCDMS will also develop channels to integrate SuperCDMS science with indigenous science education in the US.

