% !TEX root = NSF_SuperCDMS_SNOLAB_OPS.tex
\section{Broader Impacts}
\label{sec:broad}

The fundamental goal of the proposed work is to make data easily accessible to individuals who want to answer science questions.

%This work is immediately useful to the PI's own collaboration and - if the work is successful - useful to practicing scientists who can more-easily extract information from their datasets.

The immediate target audience for this work are active scientists, early-career researchers, and undergraduate students who are participating in research that require access of data for which there is not a good program.  This software will reduce the time the community spends on software development.  

The intent of this work is also to make access to science more equitable.  
\begin{itemize}
    \item High quality documentation makes the software easier to use for scientists with all levels of computing backgrounds
    \item Contribution documentation and tutorials increase the pool of contributors
    \item An online resource that provides findable materials on basic concepts in scientific computing makes it possible for novices to use the software
\end{itemize}

These aspects of the work increase science access in every case to individuals without existing background in computing.  They also increase science access to access to limited lab or institutional resources: 

\begin{itemize}
    \item Researches at undergraduate-only institutions often struggle to train undergraduates to work in a custom computing environment unless they have funding for a postdoc
    \item Lab staff and a pool of experienced graduate students and postdocs can help with undergraduate and graduate student training, but these resources are not available at many institutions.
    \item Even at institutions with significant resources for training and mentorship, access to these reasons is not always equitable.  
\end{itemize}

Large numbers of students are getting their education at community colleges and four year institutions.  Complex, poorly-documented software infrastructure that gates scientific work is an unnecessary burden on faculty at these institutions and poses a higher barrier to students because they are less likely to have access to informal computing support structure than their peers at research institutions.

The proposed work will create the kernel of an ecosystem that is usable, well-documented for both novices and experts, and enjoys the support of a community that is accessible online.  The PI believes that this infrastructure is a minimum requirement for equitable access to science analysis.



\textbf{Undergraduate Education}
Direct detection of dark matter is multidisciplinary and therefore creates a broad spectrum of opportunities for students to explore and advance in many fields. From year to year, numerous undergraduates, graduate students and postdoctoral researchers from the SuperCDMS NSF-funded institutions have benefited from these collaborative opportunities and have been guided to develop precise simulations of semiconductor-based detectors, sensitive analysis techniques, statistical applications, and application of advanced software techniques. These opportunities combined with regular mentoring have afforded many high-visibility talks for students and postdocs. 

The Physics Department at CU~Denver maintains an undergraduate-only program committed to providing students with hands-on research experience at this urban campus. The Department operates in close collaboration with the Physics Department at Metropolitan State University of Denver, and students from both institutions participate in the CU~Denver group’s activities, broadening the reach of the research program beyond a single institution. Underrepresented groups and first-generation college students are well represented in the PI’s group, and many of his mentees have proceeded to graduate school in physics and other disciplines, or embarked on careers in industry at companies such as Northrop Grumman. Currently, there are eight undergraduates in the CU~Denver group majoring in Physics, Electrical Engineering, and Mechanical Engineering.  


%\textbf{K-12 Education}
