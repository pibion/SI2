% !TEX root = NSF_SuperCDMS_SNOLAB_OPS.tex
\section{Broader Impacts}
\label{sec:broad}

The fundamental goal of the proposed work is to make data easily accessible to individuals who want to answer science questions.

This work is immediately useful to the PI's own collaboration and - if the work is successful - useful to practicing scientists who can more-easily extract information from their datasets.

But this work extends beyond the community of active scientists to scientific learners.



\textbf{Technical Development}
CDMS and SuperCDMS have developed, and continue to advance detector technologies that have significant impact both inside and outside of the dark matter community. These technologies include electrothermal feedback, transition-edge sensors (TESs); detection and utilization of athermal phonons; large, kilogram-scale phonon-mediated germanium and silicon-based detectors; arrays of large-scale, low-noise SQUIDs; and usage of LPN HEMPTs for ionization readout.

The field of direct detection of dark matter has benefited from these advances. The EDELWEISS and CRESST collaborations have both implemented some of these concepts. In addition, SuperCDMS’ pioneering of very-low-mass WIMP detection has had a strong impact on the field. There are now many R\&D efforts focused on very-low-mass WIMP detection. Multiple of these programs are run by PIs who were primarily engaged in liquid noble direct detection. The concepts developed by SuperCDMS have had far reaching impacts benefiting even direct detection experiments using the liquid nobles. 

The broader, basic-science community has benefited from SuperCDMS technology advances. The TES technology is now a core technology for multiple cosmic microwave background experiments including APEX, South Pole Telescope, and PolarBear. It is being rapidly adapted for far-infrared instruments including Super-SCUBA, single-photon spectroscopy, and large X-ray calorimetry experiments at NIST and NASA GSFC including the forthcoming Micro-X, which will launch the first TES into space. In addition, the TES technology is beginning to impact industry as TESs are being adapted for use in electron microscope sensitive surface element analysis and are being investigated for use in quantum coherence and computing.

The direct impact of SuperCDMS research is underscored by the number of graduate students impacted and the number and quality of the SuperCDMS publications. In the past three years, SuperCDMS has produced 12 Ph.D. theses and 11 science publications not including conference proceedings. The impact of these publications is underscored by their many citations (over 3,000 since 2011).

\textbf{Undergraduate, Graduate, and Postgraduate Education}
Direct detection of dark matter is multidisciplinary and therefore creates a broad spectrum of opportunities for students to explore and advance in many fields. From year to year, numerous undergraduates, graduate students and postdoctoral researchers from the SuperCDMS NSF-funded institutions have benefited from these collaborative opportunities and have been guided to develop precise simulations of semiconductor-based detectors, sensitive analysis techniques, statistical applications, and application of advanced software techniques. These opportunities combined with regular mentoring have afforded many high-visibility talks for students and postdocs. 


Sadoulet is the director of Berkeley Connect in Physics, a mentoring program within the physics department that accepts undergraduate students at all levels. The course is a small seminar class led by a physics graduate student, under the guidance of the director. The goals of the program are to help students develop understanding, community, and career preparedness that go beyond what traditional courses provide. Interactions with graduate students and faculty play a large role throughout the semester. Under Sadoulet’s leadership, this course has significantly gained in enrollment with every succeeding semester since its launch in Spring 2014. %(http://www.berkeleyconnect.berkeley.edu/physics/).

The Physics Department at CU~Denver maintains an undergraduate-only program committed to providing students with hands-on research experience at this urban campus. The Department operates in close collaboration with the Physics Department at Metropolitan State University of Denver, and students from both institutions participate in the CU~Denver group’s activities, broadening the reach of the research program beyond a single institution. Underrepresented groups and first-generation college students are well represented in the PI’s group, and many of his mentees have proceeded to graduate school in physics and other disciplines, or embarked on careers in industry at companies such as Northrop Grumman. Currently, there are eight undergraduates in the CU~Denver group majoring in Physics, Electrical Engineering, and Mechanical Engineering.  


Southern Methodist U. typically funds one to three undergraduate students to work with the Cooley group each year. In the past five years, the eight undergraduates who have participated in the group’s research have either entered graduate programs in physics or engineering or embarked on careers in industry at companies such as Texas Instruments and AT\&T.  


\textbf{K-12 Education}
SuperCDMS institutions provide strong support for K-12 education through programs for both teachers and students. Too numerous to describe in detail in this proposal, these programs take place in local schools, on campus and directly in the lab. At the core of many of the programs is engagement of students from disadvantaged backgrounds and underrepresented groups, including women. 

CU Denver, U. Florida, SMU, and USD engage middle school students in physics learning through lab tours, classroom visits and science camp. Northwestern U., UC Berkeley, U.~Florida, TAMU, SMU, and SD School of Mines and Technology support a range of high school programs, including summer research internships for students and teachers, hands-on research projects for underserved high school students of color, mini-courses to prepare Native American students for postsecondary education in science and engineering, interviews with physicists conducted by high school students, and structured physics courses for high school teachers. 
