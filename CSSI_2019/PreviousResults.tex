% !TEX root = NSF_SuperCDMS_SNOLAB_OPS.tex

\section{Results from Prior NSF Support}
\label{sec:prev-res}

In this section we describe the results from prior NSF support, divided by institution. Note that Co-PI J.~Sander has not recieved NSF Funding in the last 5 years, and that Co-PI M.~Pyle was funded through a Berkeley award as a postdoctoral researcher.

\subsection{Northwestern University, PI E. Figueroa-Feliciano}

\vspace{3pt}
\noindent\textbf{PHY-1408089, PHY-1550658}\\ 
\emph{Dark Matter and Neutrino Physics with Cryogenic Detectors}\\
Period of Support: 7/2014--8/2017, transferred from MIT to Northwestern as PHY-1550658 when Figueroa-Feliciano moved to Northwestern. Amount of Support: \$330,535\\
Publications:~\cite{SuperCDMSSensitvitiy:2016arXiv,Agnese:2015nto,OHare2015Readout-strateg,Agnese:2015ywx,Schneck2015Dark-matter-eff,Pyle2015Optimized-Desig,Agnese:2014xye,Agnese:2014vxh,2015PhRvD..91i5023B,Ruppin:2014bra,Agnese:2014aze}\\
Data products from this work are available at the \SuperCDMS collaboration's publications website~\cite{CDMSpubs} and the Figueroa group's website~\cite{FigueroaWeb}.


\subsubsection{Intellectual merit} 
This grant supports students and scientists working on an experiment that will address one of the most fundamental problems of modern science, the nature of dark matter. The \scs experiment will achieve world-leading sensitivity for dark matter searches in the 1--10~\gev mass range.

Analyses of CDMS~II data demonstrated the power of improved analysis techniques~\cite{Agnese:2014xye,Agnese:2015ywx}, and provided limits for alternate dark matter models~\cite{Agnese:2014vxh}. It also revealed a possible signal on Si~\cite{Agnese:13prl}, the analysis and publication of which was lead by the Figueroa Group, then at MIT. 
Operation of 15 \SuperCDMS ``iZIP'' detectors at Soudan since 2012 demonstrated the detectors' rejection capabilities~\cite{Agnese:13apl} and yielded world-leading sensitivity to low-mass 
DM~\cite{Agnese:13prl2,Agnese:2014aze,Agnese:2015nto}.  Our group designed the masks used to fabricate the detectors in \SuperCDMS Soudan, had major roles in the rejection capability analysis, and led the low-treshold analysis \cite{Agnese:2014aze} from \\SuperCDMS\ \Soudan.
The first operation of a single (``CDMSlite") detector with a high voltage bias provided the world's most constraining limits for DM masses below 6\,GeV$/c^2$ by achieving an extremely low energy threshold of 170\,eV electron-recoil energy~\cite{Agnese:13prl2}. A second run of the detector reached an energy threshold for electron recoils as low as 56\,eV and demonstrated the power of a fiducialization cut~\cite{Agnese:2015nto}. The Figueroa Group developed key data quality cuts for the first two CDMSlite run analyses, and is currently involved in the analysis of the third run.
 
The Figueroa Group made the first map of the so-called ``neutrino floor" in the WIMP-nucleon cross section vs WIMP mass plane~\cite{Billard:14prd}. As part of this grant the group has expanded this work to study complementarity and the effect of target on the neutrino floor~\cite{Ruppin:2014bra}, and contributed to a study on moving beyond the neutrino floor with directional detection~\cite{OHare2015Readout-strateg}. We also looked at the prospects of neutrino physics with dark matter searches~\cite{2015PhRvD..91i5023B}.
 
On the R\&D effort, the Figueroa Group has continued studying the active veto concept. As part of this grant, they developed a \geant Monte Carlo of both a bucket scintillator and Ge ring veto concepts.  They also made a design of a CDMS holder that will enable a test of a ring veto in a standard CDMS~II tower. Another R\&D effort has focused on the preparation of the ADR cryogenic system that will be made available for use for the \tunl calibration campaign described in Section~\ref{sec:tunl}. In addition, the Northwestern group has collaborated with the UF and Fermilab groups in the studies and simulations leading to the optimization and design of the calibration campaign described in this proposal.

\subsubsection{Broader impacts}
This grant strongly contributes to the training of undergraduate and graduate students and postdoctoral researchers, continuing the group's strong involvement in mentoring undergraduates from underrepresented groups by participating in the Summer Research Opportunities Program (SROP), which brings undergraduates from across the U.S. to do a summer of research at Northwestern. The group is also working to develop summer schools for training graduate students and postdocs at the ``Colegio de F\'{\i}sica Fundamental e Interdiciplinaria de las Americas" (COFI), located in San Juan, Puerto Rico.

\subsection{UC Berkeley, PI B. Sadoulet}

We summarize here the scientific and broader impact results obtained with NSF support at Berkeley over the past five years. Our group has been supported by awards PHY-1102841 (Jul 1, 2011-Jun 30, 2014, \$1,260,000, Experimental Particle Cosmology), and  PHY-1408597 (Aug 15, 2014- Jul 31 2017, \$1,177,400, Experimental Particle Cosmology). \SuperCDMS Soudan has been supported  by the NSF project award PHY-0902182 (Jul 1, 2010-Jun 30, 2013, \SuperCDMS Soudan, \$1,833,707) and operation award PHY-1004714 (Oct 2011-Sept 2013, \SuperCDMS Operation at Soudan, \$1,154,213) with UC Berkeley as the lead institution (PI B. Sadoulet) and subcontracts to MIT,  Santa Clara U., Syracuse U., U. Colorado-Denver, U. Florida, and U. Minnesota-Duluth. In addition, Berkeley (PI B. Sadoulet) has been the lead  NSF institution for the R\&D toward \SuperCDMS at SNOLAB (PHY-1242645, Sep 1, 2012- Aug 31, 2014, R\&D toward \SuperCDMS at SNOLAB, \$2,912,376) with subcontracts to MIT, Santa Clara U., and U. Colorado-Denver. Currently, Berkeley (PI B. Sadoulet) is the NSF lead institution of the \scs project (PHY-1415388, May 1, 2015-Apr 30, 2019, \scs, \$12,000,000) with subcontracts to Santa Clara U., U. Colorado-Denver, Stanford and SLAC, Texas A\&M, and U. of Minnesota-Twin Cities.

\subsubsection{Intellectual Merit}
\textbf{\SuperCDMS Soudan Science}

Over the past five years, \SuperCDMS Soudan has led the field in the search for low mass DM ($<$10 GeV/c$^{2}$) \cite{Agnese:2015nto,Agnese:2014aze,Agnese:13prl2}. Berkeley's contributions to this science output are delineated below:


\begin{compactitem}
	\item All prototype testing of the SuperCDMS Soudan iZIP for performance as well as the functional testing of every iZIP detector operated at Soudan was done in the Berkeley test facility.  
	\item Matt Pyle personally designed, tested, and optimized the SuperCDMS iZIP and 3 earlier generation prototype devices as part of his dissertation work  \cite{Pyle:12phd}. Fiducialization cuts used in \cite{Agnese:2015nto,Agnese:2014aze} were largely based on this work.
	\item Bruno Serfass led the  analysis effort during detector commissioning at Soudan and coordinates the collaboration wide software and computing group.
	% which was responsible for data processing and storage as well as the implementation of new algorithms .
%	\item The analysis effort during detector commissioning at Soudan was led by Bruno Serfass.
%	\item The Software and Computing Group which was responsible for data processing and storage as well as the implementation of new algorithms was led by Bruno Serfass. 
	\item Berkeley graduate student Todd Doughty led the underground ER/NR discrimination studies \cite{Agnese:13apl} as well as the SuperCDMS Soudan high mass dark matter analysis (to be published soon). 
	\item Berkeley led the optimization of the hardware trigger bandwidth for slower athermal phonon collection of the iZIP %due to sparse Al coverage
	 that decreased the energy threshold in our low energy analysis \cite{Agnese:2015nto}.
\end{compactitem}

\subsubsection{SuperCDMS SNOLAB Detector and Electronics R\&D}
\label{subsec:DetElRnD}
\begin{compactitem}
	\item Berkeley developed  complex impedance based techniques to distinguish the coupling mechanisms of the cryocooler noise at Soudan. Furthermore, via active measurement of the vibrational sensitivity of current athermal phonon detector technology, it has developed a vibrational specification for the SuperCDMS SNOLAB facility.

		
	\item %Charge Leakage Studies
	Dark current as a function of voltage bias has been measured for 5 test detectors at the Berkeley test facility. With this data, IR production and surface tunneling hypotheses have been disfavored.
	
	\item %Sensitivity Studies:
	The Berkeley group developed analytical detector modeling and sensitivity code for both the SuperCDMS iZIP and HV detectors that was used throughout the SuperCDMS design process (Fig. \ref{fig:SIdataProjections}) and which has recently been submitted for publication \cite{Agnese:2016cpb}.  A major upgrade to these sensitivity estimates, which is complete and soon to be published, models the quantization of ionization production which could potentially be used for ER/NR discrimination.
\item %Automation of Detector Testing:
	In CDMS-II and SuperCDMS Soudan, the labor associated with manual  testing of a detector was substantial. %has been a substantial fraction of the total detector cost. 
	As such, the SuperCDMS project has highlighted full automation of detector testing as a priority.  The Berkeley group has written semi-automated SQUID biasing and TES tuning routines %which interface with the SuperCDMS SNOLAB DAQ and warm electronics, 
	and a fully automated version should be operational in FY17. 
	
	\item %HEMT Charge Amplifier
	The SuperCDMS SNOLAB iZIP low mass sensitivity is largely determined by electronic noise in the ionization signal readout amplifier. In recent years, our collaborators at CNRS/LPN have developed low-noise, low-power high electron mobility transistors (HEMTs) with better low frequency noise performance than can be achieved with Si based JFETs. Berkeley has designed and built a HEMT-based charge amplifier located entirely at 4K to minimize the susceptibility to environmental noise pickup. HEMT switches have also been implemented to allow for the removal of the feedback resistor. 
	%without the noise of a feedback resistor. The  small spatial extent to minimize environmental noise pickups for a long  feedback loop to room temperatiure.
%	with a theoretical ionization resolution of 70 eV$_{ee}$ (300 pf input capacitance)
%	 that should also have reduced susceptibility to environmental noise due to the close proximity to the detector of all of it's components. 
We obtained a r.m.s. of  91 eV$_{ee}$ %, compared to a theoretical estimate of 70 eV$_{ee}$,
which exceeds the SuperCDMS SNOLAB goal requirement. The HEMT switch is currently under study for possible inclusion in the project baseline.
	%Substantial portions of this design are now being incorporated into the SuperCDMS SNOLAB charge amplifier designed under development by SLAC.  Berkeley is responsible for procuring HEMTs for the SuperCDMS SNOLAB Project
		
\end{compactitem}

%\subsection{\SuperCDMS Project}
\subsubsection{Broader Impacts}
Since 2004, the Berkeley group has partnered with the Level Playing Field Institute
to support the Summer Math and Science Honors (SMASH) Academy, an
enrichment program for high-achieving students of color from underserved Bay Area communities.
The group hosts a course for the incoming SMASH cohort, which engages them in research projects with
STEM grad students, many of whom are from underrepresented groups. 
The Berkeley group participates regularly
in campus and community-based outreach activities, described in more detail in Section 6.3 and is involved in the collaboration based outreach at Soudan.
%By virtue of being associated with a State Park, the Soudan Underground Laboratory has afforded
%CDMS particular access to public outreach through underground lab tours, which accommodate
%thousands of summer tourists and numerous school groups. The Lab has provided unique training
%opportunities for high school teachers, including development of related curriculum materials for
%primary and secondary school students, and undergraduate research internships. %The Soudan Education and Public Outreach program has been supported by PHY-1004714 (Oct 2011-Sept 2013, \SuperCDMS Soudan Operations, \$1,154,213) and currently by PHY-1212342 (Apr 2012-Mar 2015, Education and Public Outreach at the Soudan Underground Facility, \$267.823).

%\clearpage




\subsection{CU Denver, PI M. Huber}

%We describe here the scientific and broader impact results obtained with NSF support initiated within the past five years. In this time, the CU~Denver group has been supported by two direct awards and several sub-awards through UC Berkeley, the CDMS collaboration's lead NSF institution. 

The direct awards to CU Denver are NSF/PHY-1102795, ``SQUID-based Readout Systems for Cryogenic Dark Matter Detectors'' (\$163,711, July 1, 2011-June 30, 2015), and NSF/PHY-1408414, ``Readout Systems for Cryogenic Dark Matter Detectors'' (\$148,893, September 15, 2014-August 31, 2017). This work has resulted in publications~\cite{Agnese:13prl,Agnese:13apl,Agnese:13prl2,Agnese:2014aze,Agnese:2014vxh,Agnese:2014xye,Agnese:2015nto,Agnese:2015ywx}.
%Technical results of research and development towards (and beyond) \scs are described in Section~\ref{CUDenverResults}. Section~\ref{priorBI} describes the broader impacts of this prior NSF support.

%\smallskip
%
%
%\subsubsection{Intellectual Merit: \SuperCDMS Soudan Science}
%\label{CDMSresults}
%
%Analyses of CDMS~II data demonstrated the power of improved analysis techniques~\cite{Agnese:2014xye,Agnese:2015ywx}, revealed a possible signal on Si~\cite{Agnese:13prl}, and provided limits for alternate dark matter models~\cite{Agnese:2014vxh}. Operation of 15 \SuperCDMS ``iZIP'' detectors at Soudan since 2012 demonstrated the detectors' rejection capabilities~\cite{Agnese:13apl} and yielded world-leading sensitivity to low-mass DM~\cite{Agnese:13prl2,Agnese:2014aze,Agnese:2015nto}.  The results are in tension with WIMP DM interpretations of recent experiments~\cite{Agnese:13prl} and exclude the DM interpretation of the excess reported by the CoGeNT experiment, which uses the same target material (Ge), making comparisons between the results independent of theoretical uncertainties. First operation of a single (``CDMSlite") detector with a high voltage bias provided the world's most constraining limits for DM masses below 6\,GeV$/c^2$ by achieving an extremely low energy threshold of 170\,eV electron-recoil energy~\cite{Agnese:13prl2}. A second run of the detector reached an energy threshold for electron recoils as low as 56\,eV and demonstrated the power of a fiducialization cut~\cite{Agnese:2015nto}.
%
%
%\smallskip
%

\subsubsection{Intellectual Merit}
\label{CUDenverResults}

The CU~Denver group has played a prominent role in all aspects of the phonon readout system, utilizing the PI's extensive experience in SQUID design, characterization, and room-temperature electronics. The group supported operation of the \SuperCDMS Soudan facility through it's decomissioning, conducted research and development toward the \scs experiment, and pursued \RnD\ avenues with a view of improving detector readout for future CDMS programs. The group supports undergraduate education through extensive participation by Physics and Electrical Engineering majors in its various activities.

A primary activity of CU~Denver has been the evaluation of Superconducting Quantum Interference Device (\SQUID) series array preamplifiers. The microfabricated \SQUIDs\ are a key component of the phonon readout system. The CU~Denver group has advanced understanding of the requirements for phonon readout systems for the iZIP and HV detectors.  In collaboration with NIST-Boulder, work by the CU~Denver team contributed to the design of the next-generation SQUID array amplifier readout prototype. Intensive study of these prototypes, including measurement of the input self-inductance, dynamic resistance, equivalent input current noise, and closed-loop circuit bandwidth, allowed for a selection of the final design for the engineering detector tower. Precision measurement of the self inductances of the input and feedback coils, the mutual inductance between input and feedback coils, and the mutual inductance of each coil to the \SQUID\ fed into a model of the equivalent circuit of the TES + \SQUID\ system. Knowledge of these inductances informed design of the cold electronics striplines and connectors, by way of determining allowable levels of parasitic inductances on the input circuit. The resulting model and transfer function enabled simulation of the warm and cold electronics phonon readout system, which was essential in design of the warm-electronics circuit. These prototype tests also resulted in statistics for quantitative characteristics, qualitative metrics, and fabrication yield information. This process led to a standardized reporting format for characterizing the engineering-grade devices. To date, CU~Denver has tested all of  the $\sim$100 devices required for the Detector Tower engineering model and associated work at collaboration test facilities. The characterization has included the dc transfer function, ac (noise) performance, and measurement of other parameters of the \SQUID\ arrays.


%The CU~Denver group developed the high-volume \SQUID\ testing protocol required to meet the delivery schedule for production. The resulting infrastructure consists cryostat for measuring four devices in a single run, custom-made by the group, and electronics and software to gather the data at high resolution in an automated process.


CU~Denver has participated in the continued development of room-temperature electronics for \SQUID\ and TES readout. This participation has included measurements on the bench and with cold \SQUIDs\ of the performance of the warm electronics. These tests contributed to further development by FNAL of the Detector Control and Readout Card (DCRC) for \scs. From the testing of the prototype DCRCs, the CU~Denver group found that the noise levels and bandwidth are acceptable only if used with passive filtering. This and other testing informed the design of further generations of the warm electronics. With the FNAL and SLAC staff, the UC Denver group evaluated the schematics and design for the next-generation DCRC, which was expanded to operate 12 phonon channels (compared to the four previously), requiring a significant redesign of the circuit. The proposed circuit was evaluated to ensure it meets the technical requirements of the experiment. As a part of this evaluation, a member of the UC Denver team traveled to UC Berkeley to assist in testing the noise performance of the DCRC.


CU~Denver began studies of compatibility of the \scs\ \SQUIDs\ for planned \SuperCDMS high-voltage operation, which could require voltages that are internal to the SQUID circuit for which the devices were not designed. These studies included testing and development of instrumentation for high-voltage tests, and measurement of the voltage limit beyond which breakdown in the SQUID chip occurs. These tests contributed to the decision by the collaboration to set these high voltages by floating all of the warm electronics associated with a given detector side, so that there would be no large potential differences on an individual \SQUID\ chip.


The CU~Denver group conducted extensive tests on the performance of the \scs\ \SQUID\ arrays in various shielding environments. This testing included use of a large external high-permeability shield, close-in nested cryoperm cans, a close-in aluminum can, and a Faraday cage around the test probe. Additionally tests were done on the effects of certain metals (nickel and copper) in the vicinity of the SQUIDs, to provide information on the use of these materials in the cold-hardware design. From the tests on magnetic shielding and SQUID sensitivity, at Soudan and in Denver, the CU~Denver group found that the SQUIDs used in Soudan are sensitive to ac magnetic fields with an amplitude ($\sim$0.7 mG) similar to ambient fields found at the experiment site. The \SNOLAB\ \SQUIDs, which have a different design, were found to be less sensitive ($\sim$1.9 mG) to magnetic noise pickup. The CU~Denver group also measured the stability at cryogenic temperatures of candidate shunt resistors to be used for the TES circuit in the detector towers.

The CU~Denver group investigated the feasibility of a SQUID-based charge-pulse readout for CDMS. The resonant frequencies of the coils of a superconducting transformer, which was fabricated by the group, were measured using a vector network analyzer. The work was concluded when the use of HEMTs for charge readout, rather than \SQUIDs, was chosen.

CU~Denver supported operation of \SuperCDMS Soudan through regular shifts at the experimental site in the Soudan Underground Laboratory. The PI and the Senior Engineer both served multiple on-site shifts at Soudan. 

\subsubsection{Broader impacts}
\label{priorBI}

%Previous grants from the NSF has supported students and scientists working on an experiment addressing one of the most fundamental problems of modern science, the nature of dark matter, in a way complementary to the Large Hadron Collider and indirect detection experiments. The \scs experiment has achieved world-leading sensitivity for dark matter searches in the 1--10~\gev mass range.

The CU~Denver Physics Department maintains an undergraduate-only program committed to providing students with hands-on research experience at this urban campus. Electrical Engineering and Mechanical Engineering students have also participated in internships in the CU~Denver lab. Minorities and first-generation college students have been well represented in these internships, and many of the PI's mentees have proceeded to graduate school in physics and other disciplines, or embarked on careers in industry at companies such as Northrop Grumman and the Lockheed Martin Corporation. In the preceding five years, there have continually been six to eight undergraduates in the Huber group. 

At CU~Denver, the PI has participated with a local magnet school, the Logan School for Creative Learning, in their field trip program. Students in grades 5 - 8 have participated in small group discussions with him on topics of interest to them in the field of physics, followed by a tour of the laboratory. Their conversations with Huber and their tours of the lab are key components of presentations they give to their classmates and other visitors during the school's annual EXPO event. Students from other local middle schools and high schools have also participated in the lab's activities.

The CU~Denver PI has presented public talks on dark matter and the CDMS experiment at institutional and other local venues, which have included the CU~Denver Mini-STEM School and the Denver Cafe Scientifique. He has a standing invitation to return for future talks at Cafe Scientifique. The CU~Denver senior engineer gave a talk to a local Rotary Club about the search for dark matter and CU~Denver's involvement, addressing about 100 business and cultural leaders of the communitiy.


\subsection{\SuperCDMS Collaboration Broader Impacts}
The \SuperCDMS experimental and R\&D efforts advance phonon-mediated detectors and new active veto concepts, which have already found many applications in cosmology, astronomy and industry. 
Collaboration wide, a priority is the education of the groups' undergraduate and graduate students and postdoctoral scholars, providing them with opportunities to explore and advance in many fields. The success of this emphasis is evidenced in the high level of professional advancement of these students and postdocs, to careers as faculty, research scientists and technology experts in industry. \SuperCDMS PIs participate in programs to support K-12 students and teachers in their local schools, on campus and directly in the lab. At the core of many of the programs is engagement of students from disadvantaged backgrounds and underrepresented groups, including women. These programs include summer research internships, a program engaging underserved high school students of color in research projects, mini-courses to prepare Native American students for postsecondary education in science and engineering, and structured courses in physics for high school teachers. SuperCDMS groups have actively engaged in efforts to share \SuperCDMS science with the public, through lectures on dark matter at diverse venues, including campus lecture halls, science cafes and civic organizations. The groups regularly participate in outreach events on campus, at community festivals, science fairs and science camps. Activities include lab tours and hands-on physics demonstrations. SuperCDMS scientists have been featured in the media, including NPR's Science Friday and in the popular press. 

