% !TEX root = NSF_SuperCDMS_SNOLAB_OPS.tex

\section{Results from Prior NSF Support}
\label{sec:prev-res}

The PI has no completed NSF grants, but is a co-PI on the current SuperCDMS NSF grant entitled: ``Collaborative Research: The SuperCDMS SNOLAB Experiment (NSF-1809769).'' The award amount was split between multiple institutions and the 3-PI contingent at CU Denver was allocated \$340,000 over three award years between 8/15/2018—7/31/2021. 

This PI received 0.5 months of summer salary per year and travel funds to attend collaboration meetings.

\subsubsection{Intellectual merit} 
This grant supports students and scientists working on an experiment that addresses one of the most fundamental problems of modern science, the nature of dark matter. The \scs experiment will achieve world-leading sensitivity for dark matter searches in the 1--10~\gev mass range.

The PI's group is funded primarily for contributions to the data acquisition and data quality systems.

%\emph{Intellectual Merit:} Villano’s primary contributions to this grant are personal research time devoted to the SuperCDMS science results in Years 1 \& 2 and dissemination of these results at the collaboration meetings. 

\subsubsection{Accomplishments}
This grant has been recently awarded and there are not yet any publications.

A critical need facing the collaboration as we move to larger data sets is transitioning our analysis platform to computing clusters where we can submit jobs to batch queues.  This requires a re-working of existing analysis software, which is primarily MatLab-based and cannot be run at SLAC.

The group began testing and documenting the installation requirements of prototype python software and has since developed the first isolated build environment, allowing reliable installation of the analysis tools across platforms.

Josh Elsarboukh has worked closely with SLAC computing division to successfully deploy this analysis environment via a web interface.  This work represents an unprecedented ease of access within the collaboration and has made it possible for test facilities working on crucial R\&D and calibration efforts to efficiently analyze their data.

\subsubsection{Broader impacts}
The \SuperCDMS experimental and R\&D efforts advance phonon-mediated detectors and new active veto concepts, which have already found many applications in cosmology, astronomy and industry. 

Efforts to improve the accessibility, maintainability, and reproducibility of the analysis environment are goals that many scientific experiments share; these efforts further support the development of a scientific ecosystem that's badly needed.  Our efforts have already helped the ATLAS collaboration set up a similar environment and we are currently working to set up a similar service with ComputeCanada.

This grant supports the training of undergraduate researchers.  The CU~Denver Physics Department maintains an undergraduate-only program committed to providing students with hands-on research experience at this urban campus. Currently the PI's lab employs seven undergraduate students, all of whom are involved in developing documentation for scientific computing skills relevant to SuperCDMS.  Current students are from physics, mechanical engineering, chemistry, and computer science majors and all learn basic computing skills that are essential for scientific research.  CU~Denver enjoys a diverse student population and the PI actively seeks to represent the student population in her lab and to ensure that her lab provides a space for students to thrive.  The PI has adopted a code of conduct and published her interview process and expectations.  The PI is able to cast a broad recruiting net by teaching introductory physics and participating in college-sponsored open houses.  In addition, the PI works with students to identify conferences that are of particular interest and helps them secure funding for attendance.  Recent examples are the 2018 National Organization for the Professional Advancement of Black Chemists and Chemical Engineers (NOBCChE) conference, and the 2018 Out in STEM (oSTEM) conference.  The PI recognizes the value of the conferences as a place to practice crucial scientific communication skills and also as a place for underrepresented students to build their identities as scientists.