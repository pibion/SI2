% !TEX root = NSF_SuperCDMS_SNOLAB_OPS.tex

\section{Results from Prior NSF Support}
\label{sec:prev-res}

In this section we describe the results from prior NSF support.

\vspace{3pt}
\noindent\textbf{PHY-1809769}\\ 
\emph{Collaborative Research: The SuperCDMS SNOLAB Experiment }\\
Period of Support: 8/2018--7/2021\\
Amount of Support: \$340,000\\
Publications:~\cite{SuperCDMSSensitvitiy:2016arXiv,Agnese:2015nto,OHare2015Readout-strateg,Agnese:2015ywx,Schneck2015Dark-matter-eff,Pyle2015Optimized-Desig,Agnese:2014xye,Agnese:2014vxh,2015PhRvD..91i5023B,Ruppin:2014bra,Agnese:2014aze}\\
Data products from this work are available at the \SuperCDMS collaboration's publications website~\cite{CDMSpubs}.


\subsubsection{Intellectual merit} 
This grant supports students and scientists working on an experiment that addresses one of the most fundamental problems of modern science, the nature of dark matter. The \scs experiment will achieve world-leading sensitivity for dark matter searches in the 1--10~\gev mass range.

Analyses of CDMS~II data demonstrated the power of improved analysis techniques~\cite{Agnese:2014xye,Agnese:2015ywx}, and provided limits for alternate dark matter models~\cite{Agnese:2014vxh}. It also revealed a possible signal on Si~\cite{Agnese:13prl}, the analysis and publication of which was lead by the Figueroa Group, then at MIT. 
Operation of 15 \SuperCDMS ``iZIP'' detectors at Soudan since 2012 demonstrated the detectors' rejection capabilities~\cite{Agnese:13apl} and yielded world-leading sensitivity to low-mass 
DM~\cite{Agnese:13prl2,Agnese:2014aze,Agnese:2015nto}.  

The first operation of a single (``CDMSlite") detector with a high voltage bias provided the world's most constraining limits for DM masses below 6\,GeV$/c^2$ by achieving an extremely low energy threshold of 170\,eV electron-recoil energy~\cite{Agnese:13prl2}. A second run of the detector reached an energy threshold for electron recoils as low as 56\,eV and demonstrated the power of a fiducialization cut~\cite{Agnese:2015nto}. The Figueroa Group developed key data quality cuts for the first two CDMSlite run analyses, and is currently involved in the analysis of the third run.
 
\subsubsection{Accomplishments}
The Roberts group is funded primarily for contributions to the data acquisition and data quality systems.

A critical need facing the collaboration as we move to larger data sets is transitioning our analysis platform to computing clusters where we can submit jobs to batch queues.  This requires a re-working of existing analysis software, which is primarily MatLab-based and cannot be run at SLAC.

The group began testing and documenting the installation requirements of prototype python software and has since developed the first isolated build environment, allowing reliable installation of the analysis tools across platforms.

Josh Elsarboukh has worked closely with SLAC computing division to successfully deploy this analysis environment via a web interface.  This work represents an unprecedented ease of access within the collaboration and has made it possible for test facilities working on crucial R\&D and calibration efforts to efficiently analyze their data.

\subsubsection{Broader impacts}
The \SuperCDMS experimental and R\&D efforts advance phonon-mediated detectors and new active veto concepts, which have already found many applications in cosmology, astronomy and industry. 

This grant strongly contributes to the training of undergraduate researchers.  The CU~Denver Physics Department maintains an undergraduate-only program committed to providing students with hands-on research experience at this urban campus. Electrical Engineering and Mechanical Engineering students have also participated in internships in the CU~Denver lab. Minorities and first-generation college students have been well represented in these internships.