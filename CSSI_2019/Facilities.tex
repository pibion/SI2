%!TEX root = NSF_SuperCDMS_SNOLAB_OPS.tex

%------------------------------------------------------------------
%------------------------------------------------------------------
\clearpage
\rhead{}
\setcounter{section}{0}
\noindent\textbf{\LARGE{Facilities, Equipment, and Other Resources}}\\

\label{facilities}
This proposal  is  part of the overall \scs program, and is coordinated with other research efforts funded by NSF, DOE, and Canada. These efforts are all complimentary. \SuperCDMS Collaboration members plan to participate in this effort, and while their participation is not required to obtain the basic data products described in this proposal as these are covered by the personnel in our budget, their participation enhances the analysis and science output from the data collected. Here we list the facilities, equipment, and resources available to do the work described in this proposal, as well intended contributions from Collaborators.

\section{CUTE}
The Cryogenic Underground TEsting facility at SNOLAB is being commissioned by a team lead by our collaborators at Queen's. The facility will be operational in mid-2017. \scs will use the facility to test the pre-production \scs detectors and the first \scs HV tower. The  facility will be able to cool down and operate a full \scs tower. Apart from extra lead shielding and the readout electronics system being budgeted in this proposal, all other hardware required to run the towers in the facility will be provided and available for use. 

\section{NEXUS at Fermilab}
The Northwestern EXperimental Underground Site at Fermilab (\nexus) is being built through a collaboration between Northwestern University and Fermilab. Northwestern is supplying a new cryogen-free dilution refrigerator surrounded by a lead and poly passive shield. Fermilab is providing the site, which is 100~m underground in the MINOS tunnel, engineering support for some design work for the shields, and the installation costs of utilities and other services required to make the facility operational. The new fridge will have a base temperature of less than 10 mK and will be able to accept the 100mm SuperCDMS SNOLAB towers and detectors. The facility will be operational in late 2017. Apart from  the readout electronics system being budgeted in this proposal, all other hardware required to run the towers in the facility will be provided and available for use.

The Fermilab group, in association with other DOE-funded groups on \SuperCDMS, is planning to obtain a D-D generator for the described nuclear recoil calibration measurement. This generator would be commissioned at Fermilab and installed in the NEXUS facility by the Fermilab group.  All necessary radiation shielding and a secondary neutron “backing array” with readout, which are needed for the D-D calibration measurement, will also be supplied using existing R\&D funds that have been provided to the Fermilab \SuperCDMS group for this effort.   

\section{TUNL}

The Triangle Universities Nuclear Laboratory operates pulsed proton beam that produces a mono-energetic neutron beam. They will have 232 backing detectors by the time \SuperCDMS measurements are scheduled to be taken. The facility has performed nuclear scale calibrations with other materials and detectors before, and they are available for our use.  We have budgeted funds in this proposal to pay for beamtime at this facility.

\section{Stanford}
Stanford University will provide the small HV detectors to be used in the neutron calibration measurements described in \ref{sec:calibration}.  These detectors are fabricated as part of their DOE-sponsored \SuperCDMS R\&D program.

\section{South Dakota School of Mines, Southern Methodist University, and University of Florida}
These three institutions plan to provide postdoctoral researchers and/or students for on-site and off-site training shifts for the activities listed in this proposal, as well as further the science analysis of the data products. These personnel are to be funded through their respective pending base grants or (in the case of Mines pending a travel supplement).

\section{Northwestern}

The Northwestern group operates in a laboratory with 1500 sq ft of space. This includes a 200 sq ft clean room for cold hardware assembly. The lab has a fume hood for etching, soldering stations, chain hoist, electronic power supplies, and the measurement and control electronics commonly found in a low-temperature laboratory.

\subsection{Low Temperature facilities at Northwestern}
Northwestern operates a 35 mK base temperature Adiabatic Demagnetization Refrigerator (ADR) that is dedicated to SuperCDMS activities. It has 4 SQUID readout channels and can hold and operate several 1 cm$^2$ CDMS test devices, provided as part of SuperCDMS R\&D at Stanford. The Calibration effort described in Section~\ref{sec:calibration} of this proposal will be the sole use of this system until it is no longer needed for it.

\subsection{Radioactive Sources}
The Figueroa Lab at Northwestern has various gamma and neutron sources available for testing of detectors. These are available for use for the work described in this proposal.

\subsection{Computers}
The Figueroa SuperCDMS group maintains several computers used in operating the ADR, data analysis, and web hosting.
Group members typically work on laptops, with regular backups done to a Northwestern central computer backup site.

\subsection{Shops}
The Northwestern group has access to a highly skilled machine shop staff for fabrication (for instance of cold hardware for CDMS). The shop includes CNC milling machines and lathes and an EDM. Our graduate students and postdocs get trained in and have access to the student machine shop.

\subsection{Offices}
The Northwestern group is provided with offices for all groupmembers. They have video conferencing equipment that we routinely use for our meetings. 

\subsection{Personnel}
The Northwestern group will have two graduate students and a fraction of a postdoctoral researcher (all to be funded from their pending NSF base grant) who will be trained in part by helping in the measurements taken in this proposal and will use the data products for further scientific studies.

\section{Berkeley}
Note that Sadoulet and Pyle operate a common group, fully sharing personnel and facilities.

\subsection{Laboratory Facilities}

The UC Berkeley CDMS group operates in four adjacent laboratory rooms of 986 sq ft, 393 sq ft, 178 sq ft, and 418 sq ft. This includes a 282 sq ft clean room with a 100 sq ft anteroom for cold hardware assembly. The air quality on the assembly benches is nearly class 100. They also have access to a fume hood for etching. All cleaned parts not worked on are kept in Radon purge cabinets. 
They are currently in the process of renovating a 400 sq ft sub-basement laboratory which is ideally suited for ultra-sensitive massive detector R\&D, due to both the concrete and steel overburden of the six story Birge Hall as well as the suppressed floor vibrations. This is where the new cryogen-free dilution refrigerator described below will be installed.

\subsection{Low Temperature Facilities}
In the second of the laboratory rooms, the team operates a 75-microwatt dilution refrigerator inside a Faraday cage with a base temperature of 25 mK, which can accept the current 3" towers and 3" detectors. 
They are about to order a new cryogen-free dilution refrigerator with funds from the SuperCDMS SNOLAB project to be able to fulfill our responsibilities in SNOLAB tower testing. The new fridge will have a base temperature  of less than 10 mK and will be able to accept the 100mm SuperCDMS SNOLAB towers and detectors. The faraday cage in which it will be installed will function as a class 1000 clean room. 
The group also owns two 4K cryostats, manufactured by Infrared Labs. One of these cryostats is used for electronics R\&D, in particular the HEMTs and SQUIDs planned for SNOLAB. 

\subsection{Computers}
The Sadoulet SuperCDMS group maintains several computer nodes running with linux: two data servers totaling 13TB and one server used for web applications and user accounts management. The web server is housed at the UCB data center and utilizes the center's data recovery/back up services. The data servers also are used for data processing and analysis. Their group also maintains three MAC OS computers for detector testing and electronics design. These computing resources are maintained partly by group members and partly by the UCB College of Letters and Science UNIX/Linux consulting team. 

\subsection{Shops}
The UC Berkeley group has access to a highly skilled machine shop staff for fabrication (for instance, of cold hardware for CDMS). The shop includes CNC milling machines and lathes and two EDMs. This shop produced all of the cold hardware for the CDMS-II detector packages. In addition, group members have access to a student shop, which offers a training program.

\subsection{Offices}
The UCB group is provided with offices for Bernard Sadoulet, Matt Pyle, the research physicist and postdocs, a graduate student office and an office for the senior administrative analyst. They have video conferencing equipment, which we routinely use for our meetings. 

\subsection{Personnel}
The NSF R\&D Principal Investigator, Bernard Sadoulet,  and co PI, Matt Pyle, will each commit 25\% of their research time to this project. They not requesting any salary support in this proposal, as two months of their  summer salary are requested from the Experimental Particle Cosmology base renewal ( 3years starting August 1, 2017).

%The Pyle start-up package includes support for two graduate students who will work on the scientific tasks described in this proposal. 
The next two paragraphs describe how Bruno Serfass' and Berkeley postdoctoral effort is distributed in the various \SuperCDMS funding lines, and are included for clarification for reviewers wishing to understand what percentage of their work is devoted to what task.
%Pyle's start-up is supporting Associate Research Physicist  Bruno Serfass for his work on the SNOLAB project: 75\% from October 2016 to September 2017, and 25\% from October 2017 to September 2018. 
%%this does not add up!!!

The Berkeley group is requesting 25\% of Associate Research Physicist  Bruno Serfass' support from the NSF SuperCDMS R\&D proposal in each of the three award years (starting in July 2017) and 25\% from the Base grant renewal (3 years starting August 1, 2017). In this proposal they are requesting  33\%, 41.67\% and 50\% of his support for his contributions to Pre-operations and Commssioning for the three award years (starting in July 2017). Other funds will bring his support to 100\%.

Similarly, the Pyle start-up package supports 2x25\% of two postdocs in Federal Fiscal Years 17 and 18. Berkeley is requesting 2x25\% of the same postdocs for the separate NSF SuperCDMS R\&D proposal for the three requested award years, and a full FTE  from their base propsoal.  In this proposal they are requesting 25\%, 50\% and 75\% of a postdoc FTE for the three award years respectively. 

\section{UC Denver}
 By the time of the start of this grant period, the PI's group will have relocated to a newly renovated 1,200 sq ft research laboratory on the downtown campus of the University of Colorado Denver. This laboratory is near Physics faculty and staff offices and other Physics Department resources; and includes an 8'x8' clean room and a student office space. This laboratory is equipped with many  tools required for design and characterization of \SQUIDs\ and associated room-temperature electronics. The laboratory's robust suite of test equipment includes GPIB-controlled digital-to-analog converters, multimeters, several signal generators, five spectrum analyzers (0--100~kHz and 0.3--3000~MHz ranges), five digital oscilloscopes (100~MHz, 240~MHz, 300~MHz, 500~MHz, and 1~GHz bandwidths), four high-sensitivity (10 $\mu$V per division) analog oscilloscopes, a 1.3~GHz vector network analyzer, two lock-in amplifiers, an AC resistance bridge for precision low-impedance measurements, a remotely-controllable DC power supply, and an AC high-current power supply. Cryogenic capabilities include two storage Dewars, many cryostats for \SQUID\ tests at 4~K compatible with dipping into the storage Dewars, and two cryostats for operating superconducting devices in vacuum. The lab includes an inspection microscope, a wire-bonder, and a leak detector, as well as laboratory-fabricated room-temperature \SQUID\ electronics. This laboratory is equipped for electronics design and prototyping as well as other instrumentation development. It is a ``wet lab'' with two fume hoods, chemical storage areas (for both organics and corrosives), and is set up for three cryogenic test stations. Additionally, the lab is equipped with a large magnetic shield made of a high-$\mu$ material, a large Helmholtz coil for generating DC and AC magnetic fields, and an assortment of vacuum pumps. The lab also includes a small ``hot-work'' brazing area for cryostat fabrication.


%a $^{3}$He cryostat for measurements down to 250~mK,
%
% {\bf Clinical Facilities}
% N/A
%
%
% {\bf Animal Facilities}
% N/A


\subsection{Computer Facilities}
 Computational tools include dedicated laboratory computers and campus-level networking resources. The laboratory computers include two Windows workstations, numerous Macintosh computers, and a Linux workstation. The Windows and Macintosh computers are used electromagnetic modeling of device physical designs. The Macintosh computers are also used for device layout as well as device simulation, word processing, mechanical drawings, and data acquisition (through GPIB and TCP/IP) and analysis. The Linux workstation is used for the data acquisition system for the \scs\ experiment, which includes control of the readout electronics for that experiment. Connections between the lab and to offices are maintained by the campus Information Technology Services, as are secure web servers for all aspects of the CU~Denver campus operations, including research group activities. %The PI's laboratory web-site is maintained on this server by the group's students.


\subsection{Office Facilities}
 In addition to the student office space integral to the laboratory above, the CU~Denver group is also provided with offices for the PI and the Senior Professional Research Associate with telephone and computer network connections. 



\subsection{Other Facilities}
 Other resources available to the PI include the National Institute of Standards and Technology (NIST) Boulder Microfabrication Facility (BMF), a micro-electronics clean-room optimized for the production of superconducting, magnetic, and MEMS devices. The PI is a long-standing Research Associate of the Quantum Devices Group at NIST-Boulder and is fully qualified to use the facility and its equipment. The BMF includes an optical pattern generator for lithographic masks, a 5x reduction wafer stepper, and multiple deposition and etching systems. The PI has an office with a desktop computer and network access on the NIST site in Boulder. %The supplemental documentation includes a letter from Dr. David Rudman, manager of the facility and group leader, confirming this access.


\subsection{Major Equipment}
 The CU~Denver group operates a $^{3}$He refrigerator capable of attaining temperatures as low as 250~mK. The experimental volume is approximately 2'' in diameter and 5'' in length. This volume is sufficient for testing superconducting electronics assemblies below 4~K. % and is sufficient for the proposed scanning assembly. The CU~Denver group has a set of four vibration isolation pedestals with capacity to support the full scanning system--liquid He Dewar, $^{3}$He cryostat insert, and enclosed scanning stage. 


\subsection{Other Resources}

The CU~Denver group is provided with department-level support from a shared secretarial staff. The Physics Department machine shop, which includes a milling machine and a lathe (both CNC capable), is available to the PI's group for cryostat fabrication. A full-time machinist is available for fabrication and student certification. 



Together with parallel Base and Operations proposals this cycle, the total support requested for the PI will be 2.9 months annually. Should all proposals be fully funded, this support will be above the 2-month guideline and is justified due to a combination of CU~Denver's standard teaching load being four courses per AY and, more critically, CU~Denver Physics having neither a MasterÕs nor a Ph.D. program in Physics, requiring additional oversight of the group by the PI during the academic year. The combined AY year support from the parallel proposals (and thus a reduction in the teaching load) is necessary to allow the PI to cover specific responsibilities associated with these awards. This support above the 2-month guideline does not compensate the PI in the form of salary.



The existing \scs\ project award supports Senior Research Assistant Bruce A. Hines for his work on the \SNOLAB\ project. They are also requesting support from the NSF \SuperCDMS\ \RnD proposal in each of the three award years. The total support from all proposals, should they be fully funded, will result in 1.0 FTE in project year 1, 0.83 FTE in project year 2, and 1.0 FTE in project year 3.

%NSF SuperCDMS \RnD proposal and the
%NSF Pre-Operations and Commissioning


Funding for group members to travel to the test facilities (\SLAC, UC Berkeley) is being requested through the collaboration NSF  \SuperCDMS \RnD  proposal. Funding for group members to travel to collaboration meetings is being requested through the PI's Base award proposal.



\section{University of South Dakota}

The University of South Dakota has provided support that positively impacts this proposal'€™s potential for success by providing the PI with dedicated lab space, dedicated clean room space, and access to facilities and equipment setup as part of the Center for Ultra-low Background Experiments at Dakota collaboration including the following:

\begin{itemize}
\item
A Hall effect system for measurements at liquid nitrogen temperature and room temperature. It is made by Ecopia and measures the electronic properties of germanium crystals.
%\item
%A Photo-Thermal Ionization Spectroscopy machine that is used to measure the electrical properties of germanium crystals.
\item
An X-ray diffractometer useful for characterizing fabricated sensor properties and orienting germanium crystals.
\item
A PE-2400 RF/DC sputtering tool used to deposit sensor materials including aluminum and amorphous germanium.
\item
An OAI mask aligner used to create the precise photoresist patterns necessary for photolithographically placing sensor circuitry on a crystal.
\item
An Edwards auto 306 evaporation system used to deposit sensor materials.
\item
An Alpha step profilometer used to measure surface roughness and the thickness of deposited material.
\item
A Brewer Cee-200 photoresist dispense system in the detector development laboratory. 
\item
An RIE plasma etch tool in the detector development laboratory. 
\item
Wet-benches for acid and solvent processes used in sensor fabrication.
\item
Optical microscopes.
\end{itemize}

