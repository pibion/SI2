%!TEX root = NSF_SuperCDMS_SNOLAB_OPS.tex

%\pagenumbering{gobble}
\thispagestyle{empty}
\section*{Project Summary}
\subsection*{Elements: Improving tools based on data-description standards for gigabyte-scale data sets}
%\begin{center}
%{\large{\bf Elements: Improving tools based on data-description standards for gigabyte-scale data sets}}
%\end{center}

{\bf Overview:} The PI proposes to improve a set of data-analysis tools that require a description of the data format - rather than requiring a particular data format - to handle the gigabyte-scale datasets common in nuclear physics. The long-term goal of this work is to increase the accessibility of science analysis for both active researchers and science students.  %build and create community buy-in for a set of tools for the nuclear physics community to access binary data.  The intent is to address one piece of a common software need in the community that is repeatedly solved in isolation and that costs the entire community unnecessary time and limits access to data and science. 

In pursuit of this goal, this proposal has three objectives:  
%(1) Promote a data-description language that allows scientists to describe any binary data format.
(1) Improve existing software that provides access to data in any user-described format so that it works well with gigabyte-scale datasets.
(2) Build community adoption and support of these tools.
(3) Increase access to analysis of scientific data sets for undergraduate students, early-career researchers, and scientists who do not have access to dedicated software support. 
These objectives will be met as follows:

%{\it (1) Defining a data-description standard:} The PI will use two already-existing standards that both have broad user support, DFDL and Kaitai-Struct.  Kaitai-Struct will be used and where necessary further defined as the software created around this standard is better-suited to the needs of data analysis.

{\it (1) Improving standards-based analysis software:} Scientists who wish to analyze custom-format data can already do so with existing open-source software, e.g. kaitai-struct.  However, this software has poor performance for files larger than a gigabyte,  limiting its usefulness to the nuclear physics community.  The PI proposes to improve this software to allow responsive analysis of gigabyte-scale datasets in python.  This work is feasible thanks to libraries built and maintained by the IRIS-HEP collaboration that provide an intuitive interface to data structures optimized for rapid access to large, event-based data sets.

{\it (2) Building community:} The PI proposes yearly workshops to bring together developers and scientists for training, user feedback, and development of this software.  %In addition, the PI has requested funds for graduate and undergraduate stipends and travel awards to foster community participation.

{\it (3) Increasing Access:} To fully participate in nuclear physics research, students must have extensive scientific computing skills.
% in addition to interest and physics knowledge.  
%Physics curricula rarely prepare students for the variety of systems and skills needed to navigate typical data analysis; 
Mentor networks help some students through this maze but are not equally available.  The PI requests funding for undergraduate students to develop and test documentation and training materials for the standards-based analysis tools and for fundamental scientific computing concepts.

\textbf{Intellectual merit}: This work supports the science effort of the Super Cryogenic Dark Matter Search, an experiment that seeks to better understand the nature of dark matter.  This work supports additional high-priority science in the nuclear physics community such as the origin of the elements in the universe, fundamental symmetry testing, and radiation monitoring for national security.

\textbf{Broader impacts}: The PI proposes this work because it would be immediately useful to her own work on dark matter and because software that can provide easy access to any user-described data format would meet an urgent need of communities dealing with gigabyte-scale data.  The PI requests funds to develop and field-test documentation for this software to maximize its usefulness to the community. 

%Focusing previously-isolated software efforts on a community solution offers the benefit of software that is well-tested and well-documented and with a community that can provide support for its users.

Right now, the necessity for local development of analysis code restricts participation in science analysis.  With a small set of well-documented community tools based on data-description standards, along with open-access material introducing the computing concepts required to use those tools, the PI hopes to increase research access to a much wider group of individuals and increase the time expert scientists can spend doing science.
%With materials available to help students and new researchers learn the basic skills required for science analysis, students can participate more fully in science research, earlier.  
