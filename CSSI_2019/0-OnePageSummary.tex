%!TEX root = NSF_SuperCDMS_SNOLAB_OPS.tex

\pagenumbering{gobble}
\lhead{\Large\bf Project Summary} 

\begin{center}
{\large{\bf Elements: Improving tools based on data-description standards for gigabyte-scale data sets}}
\end{center}

{\bf Overview:} The long-term goal of the \scs collaboration is to improve the understanding of dark matter by answering the following questions: what are its constituents, what are their particle and astrophysical properties, and how do they relate to the Standard Model? In pursuit of this goal, this proposal has four objectives:  
(1) Perform necessary calibration and performance measurements of \scs detectors to maximize the science return from the experiment.
(2) Commission and operate the \scs experiment.
(3) Measure a dark matter signal or place world-leading limits for the dark matter--nucleon cross section at masses below 10~\gev.
(4) Train the next generation of scientists through research and outreach activities, and teach secondary students and the public about dark matter science. 
These objectives will be met as follows:

%\comment{Possible rewording for here and the overview page: (3) Train the next generation of scientists to be teacher scholars through engaging them in teaching secondary students and the public about dark matter science. - Joel}

{\it Experiment Characterization and Optimization:} In order to maximize the science output of \scs, detailed understanding of the response of detectors to potential signals and backgrounds is required. Measurements at two underground facilities (\nexus at Fermilab and \cute at SNOLAB) will characterize and calibrate pre-production \scs detectors and the first production \scs towers, with supporting measurements from small devices at university test facilities. Resulting data will allow energy scale calibration, measurements of background levels, understanding of the detailed phonon and electron physics of the detectors, 
development of improved background-rejection techniques,
and optimization of \scs operational parameters. 
These data may provide a dark matter search with world-leading sensitivity. 

{\it Commissioning and Operation of \scs:} This proposal will fund the NSF-portion of commissioning the \scs experiment and the 
first year of science operations and analysis, resulting in a dark matter search with world-leading sensitivity.


{\it Training and Outreach:} The activities in this proposal provide fertile ground for training of postdoctoral researchers and graduate students from the \SuperCDMS collaboration, and opportunities to share the excitement of science with the general public.

\textbf{Intellectual merit}: This grant supports an experiment that will address one of the most fundamental problems of modern science, the nature of dark matter, in a way complementary to the Large Hadron Collider and indirect detection experiments. The \scs experiment will achieve world-leading sensitivity for dark matter  masses less than 10~\gev.

\textbf{Broader impacts}: The \scs experiment will have a broad impact which extends beyond the search for dark matter. The experiment's phonon-mediated detectors have applications in cosmology, astronomy and industry. This effort will contribute opportunities for training of undergraduate and graduate students and postdoctoral researchers.

SuperCDMS will engage in education and outreach at local institutions and will coordinate our outreach at Soudan, SNOLAB, and SURF to increase the broader impact with focus on secondary education. The grant will support participating in SNOLAB-hosted Teacher Workshops, modifying and providing middle school dark matter education modules to U.S. educators, and maintaining SuperCDMS-related outreach materials for SNOLAB, Soudan, and SURF. Existing alliances will enable outreach to Sudbury First Nations communities. This effort will also support post-graduate education by providing speakers for Colegio de F\'{i}sica Fundamental e Interdiciplinaria de las \'{A}mericas summer school in Puerto Rico.

