% !TEX root = NSF_SuperCDMS_SNOLAB_OPS.tex
\section{Management}
\label{sec:management}
% RASCI chart?
% SWOT analysis?
% https://www.wrike.com/workspace.htm?acc=2707966#path=folder&id=335439022&a=2707966&c=timeline3&so=10&bso=10&sd=0&st=nt-1
% https://prod.teamgantt.com/gantt/schedule/?ids=1562286#&ids=1562286&user=&custom=&company=&hide_completed=false&date_filter=&color_filter=
% https://instagantt.com/r#

The management team of this NSF collaborative proposal will be chaired by the Northwestern PI and consist of all the PI's and Co-PI's of the proposal.  They will interact with the \nexus, \cute, and \SNOLAB facility managers to coordinate all the activities delineated in this proposal, as well as with the \SuperCDMS Collaboration for science analyses and training of students and postdocs.

The work funded by this proposal is part of the full set of \scs operations and commissioning activities, some of which are outside the scope of this proposal and are funded by DOE or Canadian funds. The management team for this proposal will be in close coordination with all parties involved in the \scs Project and Operations, as well as R\&D, and other \SuperCDMS Collaboration activities. These multi-agency collaboration-wide activities will be coordinated by a team consisting of the SuperCDMS SNOLAB Project and Operations Managers, the SuperCDMS Spokesperson, and the NSF Operations PI. This team will meet regularly to coordinate activities. 

The early operations phase of the \scs experiment covered by this proposal is when it is both necessary and appropriate to ramp up the operations management and support functions for \scs, which should all be in place for the commissioning phase of the experiment in order to ensure a smooth transition to the operations phase as quickly as possible.
%We are including a 10\% risk management overhead for the commissioning part of the proposal, as the success of commissioning will directly affect the much larger \scs Experiment. 

The detailed planning for commissioning of \scs will be led by a scientist identified to be the \scs Commissioner, with this role naturally evolving into that of a Run Coordinator during the operations phase. Details of the operations phase functions and management (which are outside the scope and period of performance of this project) are given in the \scs Experimental Operations Plan.
%Pre-operations and operations funding will be provided by DOE, NSF and NSERC/CFI. Therefore, during these phases, including commissioning of SuperCDMS SNOLAB, oversight will be provided by a Joint Oversight Group including the Program Managers from the agencies and other representatives of the DOE OHEP and NSF Physics Divisions as needed. The Operations Manager is appointed by the SLAC Director and Vice President for Research at Northwestern, with concurrence from the Fermilab and SNOLAB Directors, the DOE and NSF Program Managers, and the SuperCDMS Collaboration Spokesperson.   

%\subsection{Coordination between Operations and the Project}
%
%   There will be strong ties between the Project and Operations. The operations management will work together with the Project team to ensure that the experiment built by the project can be operated successfully and efficiently to deliver the science expected. The Project is completed upon installation of the experiment at SNOLAB. Commissioning and subsequent operation will be performed under SuperCDMS SNOLAB operations management. The pre-operations activities (mainly calibration and detector testing) that are outside the scope of the Project, but occurring at the same time as the Project, will be managed under the Operations organization, in close coordination with the Project.    
%
%Project Management and Operations Management will meet regularly to coordinate activities. Although operations activities are not part of the Project scope, some coordinated sharing of equipment and personnel can be expected. During the pre-operations phase, the Project will have the final decision on all equipment built with project funding and will have priority on use of technical personnel. The Project and Operations team will work together with the Collaboration management on sharing the effort of scientific personnel.   

\subsection{Facilities Management}
Since much of the work in this proposal occurs in dedicated underground facilities, we outline here the management of these facilities and their relationship to this grant's management.

\subsubsection{NEXUS}
The \nexus facility will receive primary management oversight from the Northwestern PI. In addition, \SuperCDMS collaborators at Fermilab will provide support to interface with Fermilab as needed. A fraction of an FTE will be provided by this grant for an operations manager, to oversee the activities delineated in this proposal. 

\subsubsection{CUTE}

The \cute facility will receive primary management oversight from our collaborators at Queen's University. This grant's management team will coordinate with Queen's to oversee the \cute activities delineated in this proposal.

\subsubsection{SNOLAB}

As mentioned above, the commissioning of \scs is an activity coordinated at the collaboration level. The PI of this grant will be part of the Operations Management team and will work closely with the \scs Operations Manager to coordinate and integrate the NSF-sponsored commissioning work with the overall multi-agency commissioning effort. \SNOLAB conducts a Gateway review process for projects which includes pre-operations activities, operation of the experiment, and decommissioning. For Gateway 1 and 2, the Project Director is the primary contact for review of the design and fabrication of the \scs experiment. For Gateway 3 and 4, the Operations Manager is the primary contact for review of the experiment pre-operations and operations. The Collaboration Canadian PI is involved in all interfaces between \scs project and operations and \SNOLAB management.  This grant's PI will be in close coordination with the Project Director, the Operations Manager and the Canadian PI to oversee the commissioning activities supported by this proposal.

%\subsection{Roles and Responsibilities of Funding Agencies}
%
%The \scs G2 Project is being jointly funded by DOE, NSF and CFI. While the scope of responsibilities for each funding agency has been carefully defined for the Project, we expect that all of these funding agencies will contribute to operating the \scs experiment, with additional research support from NSERC for Canadian researchers.  
%
%The DOE contributions to experimental operations will be managed through the SLAC National Accelerator Laboratory. The operations management office will be located there, that will report to a DOE-NSF Joint Oversight Group (JOG).


%
%\subsection{Interfaces between Operations and the Project}
%\label{sec:interfaces}
%There is obviously a strong tie between SuperCDMS Operations and the SuperCDMS SNOLAB Project, whose goal is to build the experimental facility and initial detector payload. The operations team will work together with the project team to ensure that what is built by the project will deliver the science expected. The Project is completed upon installation of the apparatus at SNOLAB. Commissioning and operation for the experiment will be handled under the SuperCDMS Operations management.
%
%There will be pre-operations activities (mainly calibration and detector testing) that are outside the scope of the Project, but occurring during the same time scale as the Project. These will be managed by the Operations organization. 
%
%\subsubsection{Management Coordination}
%The Project Management team and the Operations team will need to meet regularly to coordinate activities. By design, the operations activities are not part of the Project scope, but there may be conflicts over use of equipment or personnel. During the pre-operations phase, the Project will have the final decision on all equipment built with project funding and will have priority on use of technical personnel. The Project and Operations team will work together with the Collaboration management on sharing the effort of scientific personnel. 
%
%\subsubsection{Subsystem Management and Operations Leads}
%During the pre-operations phase, the L2 and L3 subsystem manager roles for the Project will be distinct from the Operations roles. As the Project nears completion, the Project and Operations management will discuss how best to merge expertise from the Project into Operations roles.
%
%\subsubsection{Reviews and Working Meetings}
%Project management will be invited to operations reviews and operations working meetings during the course of the Project. Similarly, operations management will be invited to project reviews and project working meetings.
%
%\subsubsection{R$\&$D}
%The collaboration will be conducting R$\&$D outside of the scope of either Project or Operations. Project management and operations management will be kept informed of the scope and resource usage for these R$\&$D activities, and the progress and results from the R$\&$D efforts.
%
%\subsubsection{Interfaces with SNOLAB}
%SNOLAB conducts a Gateway review process that parallels the DOE CD process, but also incorporates pre-operations activities, operation of the experiment and decommissioning. The Project Director is the primary contact for contact for the design and fabrication of SuperCDMS SNOLAB during Gateway 1 and 2, and the Operations Manager is the primary contact for pre-operations and operations during Gateway 3 and 4. The Collaboration Canadian lead PI is involved in all aspects of SuperCDMS SNOLAB and will be included in all interfaces with SNOLAB management. 
%