\subsection{Northwestern University}
\subsubsection{Results from Prior NSF Support}
\label{sec:prev-res}

\vspace{6pt}
\noindent\textbf{PHY-1408089, PHY-1550658}\\ 
\emph{Dark Matter and Neutrino Physics with Cryogenic Detectors}\\
Period of Support: 7/2014--8/2017, transferred from MIT to Northwestern as PHY-1550658 when Figueroa-Feliciano moved to Northwestern. Amount of Support: \$330,535\\
Publications:~\cite{SuperCDMSSensitvitiy:2016arXiv,Agnese:2015nto,OHare2015Readout-strateg,Agnese:2015ywx,Schneck2015Dark-matter-eff,Pyle2015Optimized-Desig,Agnese:2014xye,Agnese:2014vxh,2015PhRvD..91i5023B,Ruppin:2014bra,Agnese:2014aze}\\
Data products from this work are available at the \SuperCDMS collaboration's publications website~\cite{CDMSpubs} and the Figueroa group's website~\cite{FigueroaWeb}.\\






%Section~\ref{CDMSresults} summarizes the SuperCDMS science results from this support.
%Technical results of research and development towards (and beyond) SuperCDMS SNOLAB
%are described in Sections~\ref{DetectorResults}--\ref{Radonresults}.
%Section~\ref{prior_impacts} describes the broader impacts of this prior NSF support.


\textbf{Intellectual merit}: This grant supports students and scientists working on an experiment that will address one of the most fundamental problems of modern science, the nature of dark matter. The \scs experiment will achieve world-leading sensitivity for dark matter searches in the 1--10~\gev mass range.

Analyses of CDMS~II data demonstrated the power of improved analysis techniques~\cite{Agnese:2014xye,Agnese:2015ywx}, and provided limits for alternate dark matter models~\cite{Agnese:2014vxh}. It also revealed a possible signal on Si~\cite{Agnese:13prl}, the analysis and publication of which was lead by the Figueroa Group, then at MIT. 
Operation of 15 SuperCDMS ``iZIP'' detectors at Soudan since 2012 demonstrated the detectors' rejection capabilities~\cite{Agnese:13apl} and yielded world-leading sensitivity to low-mass 
DM~\cite{Agnese:13prl2,Agnese:2014aze,Agnese:2015nto}.  Our group designed the masks used to fabricate the detectors in SuperCDMS Soudan, had major roles in the rejection capability analysis, and led the low-treshold analysis \cite{Agnese:2014aze} from \SuperCDMS\ \Soudan.
The first operation of a single (``CDMSlite") detector with a high voltage bias provided the world's most constraining limits for DM masses below 6\,GeV$/c^2$ by achieving an extremely low energy threshold of 170\,eV electron-recoil energy~\cite{Agnese:13prl2}. A second run of the detector reached an energy threshold for electron recoils as low as 56\,eV and demonstrated the power of a fiducialization cut~\cite{Agnese:2015nto}. The Figueroa Group developed key data quality cuts for the first two CDMSlite run analyses, and is currently involved in the analysis of the third run.
 
 The Figueroa Group made the first map of the so-called ``neutrino floor" in the WIMP-nucleon cross section vs WIMP mass plane~\cite{Billard:14prd}. As part of this grant the group has expanded this work to study complementarity and the effect of target on the neutrino floor~\cite{Ruppin:2014bra}, and contributed to a study on moving beyond the neutrino floor with directional detection~\cite{OHare2015Readout-strateg}. We also looked at the prospects of neutrino physics with dark matter searches~\cite{2015PhRvD..91i5023B}.
 
On the R\&D effort, the Figueroa Group has continued studying the active veto concept. As part of this grant, they have developed a \geant Monte Carlo of both the bucket scintillator (Sec~\ref{sec:bucket}) and ring (Sec~\ref{sec:ring}) veto concepts. Indeed, all the results shown in Section~\ref{sec:vetoes} were done as part of the R\&D for this grant. We also made a design of a CDMS holder that will enable a test of a ring veto in a standard CDMS~II tower, which we will use for this proposal.

\textbf{Broader impacts}: The \scs experiment will have a broad impact which extends beyond the search for dark matter. This experiment is at its core a rare-event search, and its data will be used to search for a wide range of new physics. The experimental and R\&D efforts further advance phonon-mediated detectors and new active veto concepts, which have already found many applications in cosmology, astronomy and industry. This effort will strongly contribute to the training of undergraduate and graduate students and postdoctoral researchers, continuing the group's strong involvement in mentoring undergraduates from underrepresented groups by participating in the Summer Research Opportunities Program (SROP), which brings undergraduates from across the U.S. to do a summer of research at Northwestern. The group is also working to develop summer schools for training graduate students and postdocs at the ``Colegio de F\'{\i}sica Fundamental e Interdiciplinaria de las Americas" (COFI), located in San Juan, Puerto Rico.
