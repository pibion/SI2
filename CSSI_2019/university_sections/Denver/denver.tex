\subsection{CU Denver}
{\bf Overview:} The long-term goal of this project is to improve our understanding of dark matter: what it is, where it is, how it interacts with Standard Model particles, and what its role is in the evolution and fate of the universe. In pursuit of this goal, this proposal has three objectives: (1) Measure a dark matter signal or place world-leading limits for the WIMP-nucleon cross section at masses below 10~\gev. (2) Play a central role in the pre-operations and commissioning of the \scs experiment. (3) Develop and prototype new low-threshold, low-background detectors with enough sensitivity to cover all the available dark matter parameter space for masses between 1--10~\gev down to the irreducible background from solar neutrinos. These objectives will be met as follows:

{$\bullet$ \it Dark matter data analysis:} The first production \scs detectors will undergo a science run at the \cute test facility in \SNOLAB as part of the pre-operations plan. Northwestern will be actively participating in on-site data taking, off-site shifts, and data analysis activities for this data set, which will have world-leading sensitivity in the low-mass regime and give us early \scs science within the performance period of this proposal.

{$\bullet$ \it Pre-operations and Commissioning:} The Northwestern group is taking a leading role in the Pre-operation and Commissioning of \scs. This base grant will focus on calibration of the nuclear energy scale, a measurement which is essential to obtain science from the experiment, but will also support efforts in pre-production detector testing, cryogenics integration, and installation and commissioning of the experiment at \SNOLAB. 

{$\bullet$ \it Development of next-generation detectors with active veto:} The next generation of detectors for low-mass dark matter searches will require the sensitivity to explore all the available parameter space down to the neutrino floor. This requires three equally important features: low-threshold detectors, low external backgrounds, and low internal backgrounds. In collaboration with Berkeley and Texas A\&M, Northwestern will continue the development of two active-veto ideas: a ring-shaped Ge active veto detector, which will enable close to 4$\pi$ steradian active surface event vetoing, and a NaI scintillator bucket-shaped veto designed to work with the \scs Towers as a future upgrade. Both of these ideas can reduce the external electron recoil background by an order of magnitude, placing the neutrino floor within reach. In addition, Northwestern will build a prototype detector with low-threshold and electron-recoil / nuclear-recoil discrimination using NaI scintillators coupled to photon and heat sensors based on \CDMS technology.

\textbf{Intellectual merit}: This grant supports students and scientists working on an experiment that will address one of the most fundamental problems of modern science, the nature of dark matter, in a way complementary to the Large Hadron Collider and indirect detection experiments. The \scs experiment will achieve world-leading sensitivity for dark matter searches in the 1--10~\gev mass range. %This proposal will support key calibration measurements required for the scientific interpretation of the \scs data, as well as provide scientific support to the integration and commissioning of the experiment. 

\textbf{Broader impacts}: The \scs experiment will have a broad impact which extends beyond the search for dark matter. This experiment is at its core a rare-event search, and its data will be used to search for a wide range of new physics. The experimental and R\&D efforts further advance phonon-mediated detectors and new active veto concepts, which have already found many applications in cosmology, astronomy and industry. This effort will strongly contribute to the training of undergraduate and graduate students and postdoctoral researchers, continuing the group's strong involvement in mentoring undergraduates from underrepresented groups by participating in the Summer Research Opportunities Program (SROP), which brings undergraduates from across the U.S. to do a summer of research at Northwestern. The group is also working to develop summer schools for training graduate students and postdocs at the ``Colegio de F\'{\i}sica Fundamental e Interdiciplinaria de las Americas" (COFI), located in San Juan, Puerto Rico.

